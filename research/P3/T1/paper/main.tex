\documentclass[11pt,a4paper]{article}

% Packages
\usepackage[utf8]{inputenc}
\usepackage[T1]{fontenc}
\usepackage{amsmath,amssymb,amsthm}
\usepackage{mathtools}
\usepackage{geometry}
\usepackage{hyperref}

\geometry{margin=1in}

% Theorem environments
\theoremstyle{plain}
\newtheorem{theorem}{Theorem}[section]
\newtheorem{lemma}[theorem]{Lemma}
\newtheorem{proposition}[theorem]{Proposition}

\theoremstyle{definition}
\newtheorem{definition}[theorem]{Definition}
\newtheorem{remark}[theorem]{Remark}

\title{\textbf{Convexity Analysis of Energy Functional}\\\large Variational Structure in Dimensionics}
\author{P3-T1 Research Team\\Fixed-4D-Topology Project}
\date{Research Started: February 9, 2026}

\begin{document}

\maketitle

\begin{abstract}
We analyze the convexity properties of the energy functional $E(d)$ in the variational formulation of effective dimension. We prove that under suitable conditions on the coupling constants, $E(d)$ is strictly convex on $[2,4]$, ensuring unique minimizers for the variational problem.
\end{abstract}

\section{Introduction}\label{sec:intro}

The variational problem for effective dimension:
\begin{equation}\label{eq:variational}
\min_{d \in [2,4]} \left[ E(d) - T \cdot S(d) + \Lambda(d) \right]
\end{equation}
requires convexity of the energy functional for well-posedness.

\section{Energy Functional}\label{sec:energy}

\begin{definition}[Energy Functional]
The energy functional $E: [2,4] \to \mathbb{R}$ is defined as:
\begin{equation}\label{eq:energy}
E(d) = \int_M \left( R + \alpha(d-2)^2 + \beta(d-4)^2 \right) dV
\end{equation}
where:
\begin{itemize}
\item $R$ is the scalar curvature of background metric
\item $\alpha, \beta > 0$ are coupling constants
\item $M$ is the spacetime manifold
\end{itemize}
\end{definition}

\section{Convexity Analysis}\label{sec:convexity}

\subsection{Pointwise Convexity}

\begin{theorem}[Pointwise Convexity]
For the energy density $e(d) = R + \alpha(d-2)^2 + \beta(d-4)^2$:
\begin{equation}
e''(d) = 2(\alpha + \beta) > 0
\end{equation}
Therefore, $e(d)$ is strictly convex for all $\alpha, \beta > 0$.
\end{theorem}

\subsection{Integral Convexity}

\begin{theorem}[Convexity of $E(d)$]
The energy functional $E(d)$ is strictly convex on $[2,4]$:
\begin{equation}
E''(d) = 2(\alpha + \beta) \text{Vol}(M) > 0
\end{equation}
\end{theorem}

\subsection{Hessian Analysis}

For functional $E: \mathcal{D} \to \mathbb{R}$ where $\mathcal{D} = C^{\infty}(M; [2,4])$:

\begin{definition}[Second Variation]
The second variation of $E$ at $d$ in direction $\phi$ is:
\begin{equation}
\delta^2 E(d)[\phi, \phi] = \int_M 2(\alpha + \beta) \phi^2 dV
\end{equation}
\end{definition}

\begin{theorem}[Uniform Convexity]
$E$ is uniformly convex:
\begin{equation}
\delta^2 E(d)[\phi, \phi] \geq 2(\alpha + \beta) \|\phi\|_{L^2}^2
\end{equation}
\end{theorem}

\section{Entropy Contribution}\label{sec:entropy}

\begin{definition}[Entropy Functional]
The entropy is defined as:
\begin{equation}
S(d) = -\int_M d \ln d \, dV
\end{equation}
\end{definition}

\begin{proposition}[Concavity of Entropy]
$S(d)$ is strictly concave:
\begin{equation}
S''(d) = -\frac{1}{d} < 0 \quad \text{for } d > 0
\end{equation}
\end{proposition}

\section{Combined Functional}\label{sec:combined}

\begin{theorem}[Convexity of Total Functional]
The total functional:
\begin{equation}
\mathcal{F}(d) = E(d) - T \cdot S(d)
\end{equation}
is strictly convex if:
\begin{equation}
2(\alpha + \beta) + \frac{T}{d^2} > 0 \quad \forall d \in [2,4]
\end{equation}
This holds for all $\alpha, \beta, T > 0$.
\end{theorem}

\section{Research Timeline}
\begin{itemize}
\item \textbf{2026-02-09}: Research initiated
\item \textbf{2026-02-20}: Complete convexity proofs
\item \textbf{2026-02-28}: Numerical verification
\item \textbf{2026-03-05}: Paper completion
\end{itemize}

\end{document}
