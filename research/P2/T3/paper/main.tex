\documentclass[11pt,a4paper]{article}

% Packages
\usepackage[utf8]{inputenc}
\usepackage[T1]{fontenc}
\usepackage{amsmath,amssymb,amsthm}
\usepackage{mathtools}
\usepackage{geometry}
\usepackage{hyperref}

\geometry{margin=1in}

% Theorem environments
\theoremstyle{plain}
\newtheorem{theorem}{Theorem}[section]
\newtheorem{lemma}[theorem]{Lemma}
\newtheorem{proposition}[theorem]{Proposition}
\newtheorem{corollary}[theorem]{Corollary}

\theoremstyle{definition}
\newtheorem{definition}[theorem]{Definition}
\newtheorem{remark}[theorem]{Remark}

\title{\textbf{Stability Analysis of the Master Equation}\\\large Lyapunov Theory for Spectral Dimension Flow}
\author{P2-T3 Research Team\\Fixed-4D-Topology Project}
\date{Research Started: February 9, 2026}

\begin{document}

\maketitle

\begin{abstract}
We establish the Lyapunov stability of fixed points for the Master Equation governing spectral dimension flow. We prove that the UV fixed point $d_s = 2$ is exponentially stable and the IR fixed point $d_s = 4$ is also stable under the standard model $\beta(d) = -\alpha(d-2)(4-d)$. Our analysis provides rigorous foundations for the dimensional reduction mechanism in quantum gravity.
\end{abstract}

\section{Introduction}\label{sec:intro}

The Master Equation:
\begin{equation}\label{eq:master}
\mu \frac{\partial d_s}{\partial \mu} = \beta(d_s)
\end{equation}
describes the flow of spectral dimension with energy scale $\mu$. This paper establishes the stability properties of its fixed points.

\section{Setup and Definitions}\label{sec:setup}

\begin{definition}[Master Equation]
Given a $\beta$-function $\beta: [2,4] \to \mathbb{R}$, the Master Equation is:
\begin{equation}
\frac{d d_s}{d \ln \mu} = \beta(d_s)
\end{equation}
\end{definition}

\begin{definition}[Standard Model]
The standard model $\beta$-function is:
\begin{equation}\label{eq:beta}
\beta(d) = -\alpha(d-2)(4-d), \quad \alpha > 0
\end{equation}
\end{definition}

\begin{definition}[Fixed Points]
Points $d^*$ where $\beta(d^*) = 0$ are called fixed points.
\end{definition}

\section{Fixed Point Analysis}\label{sec:fixed}

\subsection{Fixed Points of Standard Model}

\begin{proposition}
The standard model \eqref{eq:beta} has exactly two fixed points:
\begin{itemize}
\item UV fixed point: $d_s^* = 2$
\item IR fixed point: $d_s^* = 4$
\end{itemize}
\end{proposition}

\begin{proof}
$\beta(d) = -\alpha(d-2)(4-d) = 0$ iff $d = 2$ or $d = 4$.
\end{proof}

\subsection{Linearization}

Let $\delta d = d_s - d^*$ be perturbation around fixed point.

\begin{equation}
\frac{d}{d\ln\mu} \delta d = \beta'(d^*) \delta d + O((\delta d)^2)
\end{equation}

\begin{proposition}[Stability Eigenvalues]
For the standard model:
\begin{align}
\beta'(2) &= 2\alpha > 0 \quad \text{(UV stable)}\\
\beta'(4) &= -2\alpha < 0 \quad \text{(IR stable)}
\end{align}
\end{proposition}

\section{Lyapunov Stability}\label{sec:lyapunov}

\subsection{Lyapunov Function}

\begin{theorem}[Lyapunov Function]
For the standard model, the function:
\begin{equation}
V(d_s) = \frac{1}{2}(d_s - 2)^2(4 - d_s)^2
\end{equation}
is a Lyapunov function for the flow.
\end{theorem}

\begin{proof}
\begin{enumerate}
\item $V(d_s) \geq 0$ with equality iff $d_s \in \{2, 4\}$
\item Along trajectories:
\begin{align}
\frac{dV}{d\ln\mu} &= \frac{\partial V}{\partial d_s} \beta(d_s)\\
&= -(d_s-2)(4-d_s)[(d_s-2)+(4-d_s)] \cdot \alpha(d_s-2)(4-d_s)\\
&= -2\alpha(d_s-2)^2(4-d_s)^2 \leq 0
\end{align}
\end{enumerate}
\end{proof}

\subsection{Exponential Stability}

\begin{theorem}[UV Exponential Stability]
The UV fixed point $d_s = 2$ is exponentially stable:
\begin{equation}
|d_s(\mu) - 2| \leq |d_s(\mu_0) - 2| \left(\frac{\mu_0}{\mu}\right)^{2\alpha}
\end{equation}
for $\mu > \mu_0$.
\end{theorem}

\section{Perturbation Analysis}\label{sec:perturbation}

\subsection{Linear Perturbation}

Consider perturbed equation:
\begin{equation}
\mu \frac{\partial d_s}{\partial \mu} = \beta(d_s) + \epsilon f(d_s, \mu)
\end{equation}

\begin{theorem}[Structural Stability]
For small $\epsilon$, the fixed points persist and maintain their stability properties.
\end{theorem}

\section{Research Timeline}
\begin{itemize}
\item \textbf{2026-02-09}: Research initiated
\item \textbf{2026-02-15}: Complete Lyapunov analysis
\item \textbf{2026-02-20}: Perturbation theory finalized
\item \textbf{2026-02-25}: Paper completion
\end{itemize}

\end{document}
