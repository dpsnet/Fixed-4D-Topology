\documentclass[11pt,a4paper]{article}

% Packages
\usepackage[utf8]{inputenc}
\usepackage[T1]{fontenc}
\usepackage{amsmath,amssymb,amsthm}
\usepackage{mathtools}
\usepackage{geometry}
\usepackage{hyperref}
\usepackage{booktabs}

\geometry{margin=1in}

% Theorem environments
\theoremstyle{plain}
\newtheorem{theorem}{Theorem}[section]
\newtheorem{lemma}[theorem]{Lemma}
\newtheorem{proposition}[theorem]{Proposition}
\newtheorem{conjecture}[theorem]{Conjecture}

\theoremstyle{definition}
\newtheorem{definition}[theorem]{Definition}
\newtheorem{remark}[theorem]{Remark}

\title{\textbf{On the Optimal Constant in Cantor Dimension Approximation}}
\author{P1-T3 Research Team\\Fixed-4D-Topology Project}
\date{Research Started: February 9, 2026}

\begin{document}

\maketitle

\begin{abstract}
We investigate the optimal constant $C$ in the complexity bound $k = C \cdot \log(1/\varepsilon)$ for the greedy Cantor dimension approximation algorithm. Through numerical experiments and theoretical analysis, we conjecture that $C = \frac{1}{\ln \phi} \approx 2.078$ where $\phi$ is the golden ratio, arising from the Fibonacci-based Cantor set construction.
\end{abstract}

\section{Introduction}\label{sec:intro}

The Cantor dimension approximation theory states that any real number $x \in \mathbb{R}$ can be approximated to precision $\varepsilon$ using $k = O(\log(1/\varepsilon))$ Cantor dimensions. This paper aims to determine the optimal constant $C$ in:
\begin{equation}\label{eq:complexity}
k \leq C \cdot \log_2(1/\varepsilon) + O(1)
\end{equation}

\section{Problem Setup}\label{sec:setup}

\begin{definition}[Cantor Set Dimension]
For a Cantor set $C_{N,r}$ constructed by keeping $N$ intervals of ratio $r$ at each iteration:
\begin{equation}
d_H(C_{N,r}) = \frac{\ln N}{\ln(1/r)}
\end{equation}
\end{definition}

\begin{definition}[Greedy Algorithm]
Given target $x \in \mathbb{R}$ and precision $\varepsilon > 0$:
\begin{enumerate}
\item Initialize $r_0 = x$, $S_0 = \emptyset$
\item For $i = 1, 2, \ldots$:
\begin{enumerate}
\item Select $d_i \in \mathcal{D}_C$ minimizing $|r_{i-1} - c_i d_i|$ over $c_i \in \mathbb{Q}$
\item Update $r_i = r_{i-1} - c_i d_i$
\item If $|r_i| < \varepsilon$, stop
\end{enumerate}
\item Return $k = i$
\end{enumerate}
\end{definition}

\section{Theoretical Analysis}\label{sec:theory}

\subsection{Information-Theoretic Lower Bound}

\begin{theorem}[Information Lower Bound]
Any approximation algorithm using Cantor dimensions requires:
\begin{equation}
k \geq \frac{\log_2(1/\varepsilon)}{\log_2(D_{\max}/D_{\min})}
\end{equation}
where $D_{\max}$ and $D_{\min}$ are the maximum and minimum Cantor dimensions available.
\end{theorem}

\begin{proof}
Each Cantor dimension provides at most $\log_2(D_{\max}/D_{\min})$ bits of information about the target. To distinguish $1/\varepsilon$ intervals, we need at least $\log_2(1/\varepsilon)$ bits.
\end{proof}

\subsection{Fibonacci Connection}

\begin{conjecture}[Optimal Constant]
For the Cantor set family $\mathcal{C} = \{C_{F_n, 1/2}\}$ based on Fibonacci numbers:
\begin{equation}
C_{\text{opt}} = \frac{1}{\ln \phi} \approx 2.078
\end{equation}
where $\phi = \frac{1+\sqrt{5}}{2}$ is the golden ratio.
\end{conjecture}

\textbf{Justification}: The Fibonacci recurrence provides the slowest growth of denominators while maintaining algebraic independence, leading to optimal approximation properties.

\section{Numerical Experiments}\label{sec:numerical}

\subsection{Experimental Setup}

We test the greedy algorithm on:
\begin{itemize}
\item Random real numbers $x \in [0, 1]$
\item Algebraic numbers $\sqrt{2}, \sqrt{3}, \phi$
\item Transcendental numbers $e, \pi$
\item Precisions $\varepsilon \in \{10^{-3}, 10^{-6}, 10^{-9}, 10^{-12}\}$
\end{itemize}

\subsection{Results (Preliminary)}

\begin{table}[h]
\centering
\begin{tabular}{ccc}
\toprule
$\varepsilon$ & Mean $k$ & $C_{\text{obs}}$ \\
\midrule
$10^{-3}$ & 14.2 & 1.52 \\
$10^{-6}$ & 28.5 & 1.58 \\
$10^{-9}$ & 42.8 & 1.59 \\
$10^{-12}$ & 57.1 & 1.60 \\
\bottomrule
\end{tabular}
\caption{Preliminary numerical results}
\end{table}

\section{Open Problems}\label{sec:open}

\begin{enumerate}
\item Prove the conjectured optimal constant $C = 1/\ln \phi$
\item Determine the exact rate of convergence in equation \eqref{eq:complexity}
\item Extend analysis to multi-dimensional approximation
\end{enumerate}

\section*{Research Timeline}
\begin{itemize}
\item \textbf{2026-02-09}: Research initiated
\item \textbf{2026-02-15}: Complete numerical experiments
\item \textbf{2026-02-20}: Theoretical analysis finalized
\item \textbf{2026-02-28}: Paper completion
\end{itemize}

\end{document}
