% Chapter 2: Mathematical Overview

This section establishes notation and reviews essential background material for the unified dimensionics framework.

\subsection{Notation and Conventions}

We work primarily in separable metric spaces $(X,d)$. Key notation:
\begin{itemize}
\item $\dH$: Hausdorff dimension
\item $\dB$: Box-counting dimension  
\item $\ds$: Spectral dimension
\item $\deff$: Effective dimension
\item $\mathcal{H}^s$: $s$-dimensional Hausdorff measure
\item $\dim_{top}$: Topological dimension
\item $C(F)$: Cantor set with ratio $F$
\item $SG$: Sierpinski gasket
\end{itemize}

\subsection{Classical Dimension Theories}

\subsubsection{Hausdorff Dimension}

\begin{definition}[Hausdorff Dimension]
For $F \subset \mathbb{R}^n$, the Hausdorff dimension is:
$$\dH(F) = \inf\{s : \mathcal{H}^s(F) = 0\} = \sup\{s : \mathcal{H}^s(F) = \infty\}$$
\end{definition}

The Hausdorff dimension is the most widely used fractal dimension, satisfying:
\begin{itemize}
\item Countable stability: $\dH(\bigcup_i F_i) = \sup_i \dH(F_i)$
\item Monotonicity: $E \subseteq F \Rightarrow \dH(E) \leq \dH(F)$
\item Lipschitz invariance under bi-Lipschitz maps
\end{itemize}

\subsubsection{Box-Counting Dimension}

\begin{definition}[Box-Counting Dimension]
The lower and upper box dimensions are:
$$\underline{\dB}(F) = \liminf_{\epsilon \to 0} \frac{\log N_\epsilon(F)}{-\log \epsilon}, \quad
\overline{\dB}(F) = \limsup_{\epsilon \to 0} \frac{\log N_\epsilon(F)}{-\log \epsilon}$$
where $N_\epsilon(F)$ is the minimum number of $\epsilon$-balls covering $F$.
\end{definition}

When the limits coincide, we denote the common value by $\dB(F)$.

\subsection{The Dimension Hierarchy}

For any compact metric space $F$, the following inequalities hold:
$$\dim_{top}(F) \leq \ds(F) \leq \dH(F) \leq \underline{\dB}(F) \leq \overline{\dB}(F)$$

These inequalities reflect fundamental structural constraints that any unified theory must respect.

\subsection{Standard Fractals}

\begin{table}[htbp]
\centering
\caption{Dimensions of Standard Fractals}
\begin{tabular}{lcccc}
\toprule
\textbf{Fractal} & $\dim_{top}$ & $\ds$ & $\dH$ & $\dB$ \\
\midrule
Cantor set & 0 & N/A & 0.631 & 0.631 \\
Sierpinski gasket & 1 & 1.365 & 1.585 & 1.585 \\
Koch curve & 1 & N/A & 1.262 & 1.262 \\
\bottomrule
\end{tabular}
\end{table}

\subsection{Function Spaces on Fractals}

\begin{definition}[Besov Space on Fractal]
For a $d$-set $F \subset \mathbb{R}^n$, the Besov space $B^s_{p,q}(F)$ is defined via traces:
$$B^s_{p,q}(F) = \{f|_F : f \in B^s_{p,q}(\mathbb{R}^n)\}$$
with quotient norm.
\end{definition}

These spaces provide the analytic foundation for studying differential equations on fractals.
