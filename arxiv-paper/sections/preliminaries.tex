\subsection{Cantor Class Fractals}

\begin{definition}[Generalized Cantor Set]
\label{def:generalized_cantor}
Let $r \in (0, 1/2)$ be a scaling parameter. The \emph{generalized Cantor set} $\Cantor_r \subseteq [0,1]$ is defined as the unique non-empty compact set satisfying:
\begin{equation}
\Cantor_r = r \cdot \Cantor_r \cup (r \cdot \Cantor_r + (1-r))
\end{equation}
Equivalently, $\Cantor_r$ is the attractor of the iterated function system $\{f_1, f_2\}$ where $f_1(x) = rx$ and $f_2(x) = rx + (1-r)$.
\end{definition}

\begin{definition}[Cantor Class Dimensions]
\label{def:cantor_class}
The \emph{Cantor class} $\mathcal{D}_{\Cantor}$ is the set of Hausdorff dimensions of generalized Cantor sets:
\begin{equation}
\mathcal{D}_{\Cantor} = \left\{ \dimH(\Cantor_r) : r \in (0, 1/2) \cap \Q \right\}
\end{equation}
Since $\dimH(\Cantor_r) = \frac{\log 2}{\log(1/r)}$, we have:
\begin{equation}
\mathcal{D}_{\Cantor} = \left\{ \frac{\log 2}{\log(1/r)} : r \in (0, 1/2) \cap \Q \right\}
\end{equation}
\end{definition}

\begin{example}[Standard Cantor Set]
For $r = 1/3$, we obtain the standard middle-thirds Cantor set:
\begin{equation}
\dimH(\Cantor_{1/3}) = \frac{\log 2}{\log 3} \approx 0.6309297536
\end{equation}
\end{example}

\begin{definition}[Rational Combinations]
\label{def:rational_combinations}
For a finite subset $\{d_1, \ldots, d_k\} \subseteq \mathcal{D}_{\Cantor}$, the set of \emph{rational combinations} is:
\begin{equation}
\mathcal{R}(d_1, \ldots, d_k) = \left\{ \sum_{i=1}^{k} q_i d_i : q_i \in \Q \right\}
\end{equation}
The set of all rational combinations from $\mathcal{D}_{\Cantor}$ is:
\begin{equation}
\mathcal{R}_{\Cantor} = \bigcup_{k=1}^{\infty} \bigcup_{\{d_1, \ldots, d_k\} \subseteq \mathcal{D}_{\Cantor}} \mathcal{R}(d_1, \ldots, d_k)
\end{equation}
\end{definition}

\subsection{Linear Independence and Dimension}

\begin{definition}[Linear Independence over $\Q$]
\label{def:linear_independence}
A set $S \subseteq \R$ is \emph{linearly independent over $\Q$} if for any finite subset $\{s_1, \ldots, s_n\} \subseteq S$ and any $q_1, \ldots, q_n \in \Q$:
\begin{equation}
\sum_{i=1}^{n} q_i s_i = 0 \implies q_1 = q_2 = \cdots = q_n = 0
\end{equation}
\end{definition}

\begin{lemma}[Cardinality Argument]
\label{lem:cardinality}
Let $S \subseteq \R$ be a countable set. Then the set of finite rational combinations of elements from $S$ has cardinality $\aleph_0$ (countable), while $\R$ has cardinality $2^{\aleph_0}$ (uncountable).
\end{lemma}

\begin{proof}
The set of finite rational combinations can be written as:
\begin{equation}
\mathcal{C} = \bigcup_{n=1}^{\infty} \left\{ \sum_{i=1}^{n} q_i s_i : q_i \in \Q, s_i \in S \right\}
\end{equation}
For each $n$, the set of $n$-term combinations is in bijection with $\Q^n \times S^n$, which is countable. A countable union of countable sets is countable. Therefore $|\mathcal{C}| = \aleph_0 < 2^{\aleph_0} = |\R|$.
\end{proof}

\Cref{lem:cardinality} shows why exact representation of all real numbers using countable fractal dimensions is impossible. This motivates our focus on approximation.

\subsection{Approximation Framework}

\begin{definition}[Approximation Problem]
\label{def:approx_problem}
Given a target $\alpha \in \R$ and precision $\eps > 0$, find $d \in \mathcal{R}_{\Cantor}$ such that:
\begin{equation}
|\alpha - d| < \eps
\end{equation}
The \emph{complexity} of the approximation is the minimum number of Cantor dimensions needed to achieve the error bound.
\end{definition}

\begin{definition}[Greedy Approximation]
\label{def:greedy_approx}
For target $\alpha$ and precision $\eps$, the greedy algorithm proceeds as follows:
\begin{enumerate}[leftmargin=*]
    \item Initialize residual $r_0 = \alpha$
    \item At step $k$: choose $d_{i_k} \in \mathcal{D}_{\Cantor}$ and $c_k \in \Q$ minimizing $|r_{k-1} - c_k \cdot d_{i_k}|$
    \item Update: $r_k = r_{k-1} - c_k \cdot d_{i_k}$
    \item Stop when $|r_k| < \eps$
\end{enumerate}
\end{definition}

We denote the approximation after $k$ steps by:
\begin{equation}
A_k(\alpha) = \sum_{j=1}^{k} c_j \cdot d_{i_j}
\end{equation}
