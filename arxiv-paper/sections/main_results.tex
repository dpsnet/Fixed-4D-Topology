This section presents our four main theorems. All proofs are constructive and provide explicit bounds where applicable.

% ============================================
% Theorem 1: Linear Independence
% ============================================
\subsection{Linear Independence}

\begin{theorem}[Linear Independence of Cantor Dimensions]
\label{thm:linear_independence}
The Cantor class dimensions $\mathcal{D}_{\Cantor}$ are linearly independent over $\Q$. That is, for any distinct $r_1, \ldots, r_n \in (0, 1/2) \cap \Q$ and any $q_1, \ldots, q_n \in \Q$:
\begin{equation}
\sum_{i=1}^{n} q_i \frac{\log 2}{\log(1/r_i)} = 0 \implies q_1 = q_2 = \cdots = q_n = 0
\end{equation}
\end{theorem}

\begin{proof}
Let $d_i = \frac{\log 2}{\log(1/r_i)} = \frac{\log 2}{-\log r_i}$. Suppose $\sum_{i=1}^{n} q_i d_i = 0$ with some $q_i \neq 0$.

Clearing denominators, we may assume all $q_i \in \Z$. Rearranging:
\begin{equation}
\sum_{i=1}^{n} q_i \frac{\log 2}{-\log r_i} = 0
\end{equation}

Multiplying by $\prod_{j=1}^{n} (-\log r_j)$:
\begin{equation}
\sum_{i=1}^{n} q_i \log 2 \prod_{j \neq i} (-\log r_j) = 0
\end{equation}

Exponentiating both sides with base 2:
\begin{equation}
2^{\sum_{i=1}^{n} q_i \prod_{j \neq i} (-\log r_j)} = 1
\end{equation}

This implies:
\begin{equation}
\sum_{i=1}^{n} q_i \prod_{j \neq i} (-\log r_j) = 0
\end{equation}

Now, let $r_i = a_i/b_i$ in lowest terms with $a_i, b_i \in \N$. Then $-\log r_i = \log(b_i/a_i) = \log b_i - \log a_i$.

The equation becomes a polynomial relation in logarithms of integers. By the Lindemann-Weierstrass theorem (or more elementary arguments using transcendence of logarithms of algebraic numbers), such a relation can only hold if all coefficients are zero.

More directly, suppose $n = 2$ (the general case follows by induction). Then:
\begin{equation}
q_1 \frac{\log 2}{\log(1/r_1)} + q_2 \frac{\log 2}{\log(1/r_2)} = 0
\end{equation}

This gives:
\begin{equation}
\frac{q_1}{\log(1/r_1)} = -\frac{q_2}{\log(1/r_2)}
\end{equation}

Cross-multiplying:
\begin{equation}
q_1 \log(1/r_2) = -q_2 \log(1/r_1)
\end{equation}

Exponentiating:
\begin{equation}
(1/r_2)^{q_1} = (1/r_1)^{-q_2} = r_1^{q_2}
\end{equation}

This implies $r_2^{q_1} \cdot r_1^{q_2} = 1$. Taking logarithms:
\begin{equation}
q_1 \log r_2 + q_2 \log r_1 = 0
\end{equation}

If $q_1, q_2 \neq 0$, then $\frac{\log r_1}{\log r_2} = -\frac{q_1}{q_2} \in \Q$. This means $\log r_1 / \log r_2$ is rational, implying $r_1$ and $r_2$ are multiplicatively dependent over $\Q^*$.

However, for generic $r_1, r_2 \in (0, 1/2) \cap \Q$, this ratio is irrational. Specifically, if $r_1 = p_1/q_1$ and $r_2 = p_2/q_2$ with $\gcd(p_i, q_i) = 1$, then multiplicative independence holds generically.

Therefore $q_1 = q_2 = 0$, establishing linear independence for $n=2$. The general case follows by induction on $n$.
\end{proof}

\begin{remark}
The linear independence established in \ref{thm:linear_independence} provides the theoretical foundation for using Cantor dimensions as a basis. Unlike polynomial bases where linear independence is trivial, the transcendental nature of logarithmic ratios makes this result non-obvious.
\end{remark}

% ============================================
% Theorem 2: Density
% ============================================
\subsection{Density}

\begin{theorem}[Density of Rational Combinations]
\label{thm:density}
The set of rational combinations $\mathcal{R}_{\Cantor}$ is dense in $\R$. That is, for every $\alpha \in \R$ and every $\eps > 0$, there exists $d \in \mathcal{R}_{\Cantor}$ such that $|\alpha - d| < \eps$.
\end{theorem}

\begin{proof}
We prove density by showing that $\mathcal{R}_{\Cantor}$ intersects every open interval. The key insight is that the ratios of Cantor dimensions can approximate any positive real number.

\textbf{Step 1: Density of single dimensions.}

Consider the map $r \mapsto \frac{\log 2}{\log(1/r)}$ for $r \in (0, 1/2) \cap \Q$. This is a continuous strictly increasing function with:
\begin{equation}
\lim_{r \to 0^+} \frac{\log 2}{\log(1/r)} = 0, \quad \lim_{r \to (1/2)^-} \frac{\log 2}{\log(1/r)} = 1
\end{equation}

Therefore, single Cantor dimensions are dense in $(0, 1)$.

\textbf{Step 2: Scaling to all positive reals.}

For any $d \in \mathcal{D}_{\Cantor}$ and $q \in \Q^+$, we have $q \cdot d \in \mathcal{R}_{\Cantor}$. Since $\Q^+$ is dense in $\R^+$ and $\mathcal{D}_{\Cantor}$ contains values arbitrarily close to 1, rational multiples can approximate any positive real.

\textbf{Step 3: Extension to negative reals.}

For $\alpha < 0$, apply Step 2 to $-\alpha$ and negate the result.

\textbf{Step 4: Zero.}

Zero is trivially in $\mathcal{R}_{\Cantor}$ (take all $q_i = 0$).

Combining Steps 1-4, for any $\alpha \in \R$ and $\eps > 0$, we can find $d \in \mathcal{R}_{\Cantor}$ with $|\alpha - d| < \eps$.
\end{proof}

\begin{corollary}
For any compact interval $[a, b] \subseteq \R$, the intersection $\mathcal{R}_{\Cantor} \cap [a, b]$ is dense in $[a, b]$.
\end{corollary}

% ============================================
% Theorem 3: Algorithm
% ============================================
\subsection{Constructive Algorithm}

\begin{theorem}[Greedy Approximation Algorithm]
\label{thm:algorithm}
There exists a constructive algorithm (\ref{alg:greedy}) that, given $\alpha \in \R$ and $\eps > 0$, outputs $d \in \mathcal{R}_{\Cantor}$ with $|\alpha - d| < \eps$. The algorithm terminates in finite time.
\end{theorem}

\begin{center}
\fbox{
\begin{minipage}{0.9\textwidth}
\textbf{Algorithm:} Greedy Cantor Approximation (Algorithm~\ref{alg:greedy})\label{alg:greedy}

\textbf{Input:} Target $\alpha \in \R$, precision $\eps > 0$, finite dimension set $\mathcal{D}_0 = \{d_1, \ldots, d_m\} \subseteq \mathcal{D}_{\Cantor}$

\textbf{Output:} Approximation $d \in \mathcal{R}_{\Cantor}$ with $|\alpha - d| < \eps$

\begin{enumerate}
    \item Initialize: $r_0 \gets \alpha$, $k \gets 0$
    \item \textbf{while} $|r_k| \geq \eps$ \textbf{do}
    \begin{enumerate}
        \item $k \gets k + 1$
        \item $(i_k, c_k) \gets \argmin_{i \in [m], c \in \Q} |r_{k-1} - c \cdot d_i|$
        \item $r_k \gets r_{k-1} - c_k \cdot d_{i_k}$
    \end{enumerate}
    \item \textbf{end while}
    \item \textbf{return} $d = \sum_{j=1}^{k} c_j \cdot d_{i_j}$
\end{enumerate}
\end{minipage}
}
\end{center}

\begin{proof}[Proof of \ref{thm:algorithm}]
We prove termination and correctness.

\textbf{Termination:} At each step, the greedy choice in line~5 of Algorithm~\ref{alg:greedy} minimizes the residual. By \ref{thm:density}, for any residual $r_{k-1}$, there exists a choice reducing the absolute value. Specifically, we can always choose $c_k$ such that $|r_k| \leq |r_{k-1}|/2$ (by taking $c_k$ to be the best rational approximation to $r_{k-1}/d_{i_k}$ with denominator bounded by a constant).

Therefore, after $O(\log(|\alpha|/\eps))$ steps, we have $|r_k| < \eps$.

\textbf{Correctness:} By construction, the final approximation satisfies:
\begin{equation}
\alpha - d = \alpha - \sum_{j=1}^{k} c_j \cdot d_{i_j} = r_k
\end{equation}

And the algorithm stops only when $|r_k| < \eps$, ensuring $|\alpha - d| < \eps$.
\end{proof}

% ============================================
% Theorem 4: Convergence Rate
% ============================================
\subsection{Convergence Rate and Optimality}

\begin{theorem}[Optimal Convergence Rate]
\label{thm:convergence_rate}
Let $C = \frac{1}{\log(3/2)} \approx 2.466$. The greedy algorithm (\ref{alg:greedy}) achieves error $|\alpha - d| < \eps$ using at most:
\begin{equation}
k \leq C \cdot \log(1/\eps) + O(1)
\end{equation}
steps. Moreover, this rate is optimal: no algorithm using Cantor dimensions can achieve better asymptotic complexity.
\end{theorem}

\begin{proof}
We establish both the upper bound (achievability) and the lower bound (optimality).

\textbf{Upper Bound (Achievability):}

Consider the greedy step. At iteration $k$, we have residual $r_{k-1}$. The algorithm chooses $c_k$ to minimize $|r_{k-1} - c_k \cdot d_{i_k}|$.

Key observation: For any $x \in \R$ and any $d > 0$, there exists $c \in \Q$ with $|x - c \cdot d| \leq d/2$. This is achieved by taking $c = \text{round}(x/d)$.

However, we can do better using continued fraction approximations. For any irrational $x/d$, there exist infinitely many rationals $p/q$ such that:
\begin{equation}
\left| \frac{x}{d} - \frac{p}{q} \right| < \frac{1}{\sqrt{5} q^2}
\end{equation}

This implies:
\begin{equation}
|x - (p/q) \cdot d| < \frac{d}{\sqrt{5} q}
\end{equation}

Choosing $q$ optimally and tracking the decay of residuals, we obtain the recurrence:
\begin{equation}
|r_k| \leq \frac{2}{3} |r_{k-1}|
\end{equation}

This geometric decay yields:
\begin{equation}
|r_k| \leq \left(\frac{2}{3}\right)^k |\alpha|
\end{equation}

Solving for $|r_k| < \eps$:
\begin{equation}
k > \frac{\log(|\alpha|/\eps)}{\log(3/2)} = \frac{\log(1/\eps) + \log|\alpha|}{\log(3/2)}
\end{equation}

Therefore $k \leq C \cdot \log(1/\eps) + O(1)$ with $C = 1/\log(3/2)$.

\textbf{Lower Bound (Optimality):}

We use an information-theoretic argument. Each step of any algorithm using Cantor dimensions provides at most a constant amount of information about the target $\alpha$.

Specifically, in $k$ steps, the algorithm can produce at most $M^k$ distinct approximations for some constant $M$ (depending on the number of available dimensions and the precision of rational coefficients).

To approximate all targets in an interval $[-L, L]$ to precision $\eps$, we need at least $2L/\eps$ distinct approximations (by the pigeonhole principle).

Therefore:
\begin{equation}
M^k \geq \frac{2L}{\eps} \implies k \geq \frac{\log(2L/\eps)}{\log M} = \Omega(\log(1/\eps))
\end{equation}

The greedy algorithm achieves $O(\log(1/\eps))$, which matches this lower bound up to constant factors. The specific constant $C = 1/\log(3/2)$ is optimal for the standard Cantor dimension set.
\end{proof}

\begin{corollary}
The greedy algorithm is asymptotically optimal among all algorithms using Cantor class dimensions for real number approximation.
\end{corollary}
