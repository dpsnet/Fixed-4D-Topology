% Chapter 1: Introduction with Historical Terminology Clarification
\section{Introduction}
\label{sec:introduction}

\subsection{Historical Evolution and Clarification of Terminology}
\label{subsec:terminology_history}

The phenomenon central to this review---the scale-dependent change in a certain mathematical parameter characterizing dynamical systems---has been described in the literature using various terminologies that have evolved over time, leading to considerable conceptual confusion. To establish a precise framework, we must first clarify the historical development of key terms and distinguish carefully between mathematical definitions, physical interpretations, and popular descriptions.

\subsubsection{Origins: Spectral Geometry (1949--1965)}

The mathematical foundation was laid by Minakshisundaram and Pleijel in 1949 \cite{Minakshisundaram1949}, who introduced the asymptotic expansion of the heat kernel trace:
\begin{equation}
K(t) \sim \frac{1}{(4\pi t)^{d/2}}\sum_{k=0}^{\infty} a_k t^k
\label{eq:mp_expansion}
\end{equation}

In this original context, the exponent $d/2$ was simply half the topological dimension of the manifold. The term "spectral dimension" did not appear; rather, mathematicians spoke of the "asymptotic behavior of the spectrum" or the "Weyl asymptotics." The dimension $d$ was unambiguously the topological dimension of the space.

DeWitt's 1965 work on quantum field theory in curved spacetime \cite{DeWitt1965} used the heat kernel for calculating effective actions, but always with the understanding that the underlying spacetime dimension was fixed. The heat kernel coefficients ($a_0$, $a_1$, $a_2$) were geometric invariants of a fixed-dimensional manifold.

\subsubsection{Introduction of ``Spectral Dimension'' (1990s)}

The term ``spectral dimension'' ($d_s$) emerged in the study of fractal geometries and anomalous diffusion, where it was defined as:
\begin{equation}
d_s = -2 \lim_{t\to\infty} \frac{\ln K(t)}{\ln t}
\label{eq:spectral_def_fractal}
\end{equation}

**Critical distinction**: For fractals, the spectral dimension naturally differs from the topological (Hausdorff) dimension because fractals themselves have non-integer dimension. The spectral dimension provided a measure of how diffusion processes ``sample'' the fractal structure. There was no implication that space itself changed dimension; rather, different measures of ``dimension'' (Hausdorff, box-counting, spectral) captured different aspects of the fractal's geometry.

\subsubsection{The Terminological Shift in Quantum Gravity (2005)}

The pivotal development came with Causal Dynamical Triangulations (CDT). In their 2005 paper, Ambjørn, Jurkiewicz, and Loll \cite{Ambjorn2005} wrote:

\begin{quote}
``...the spectral dimension at short distances... \textbf{appears to be} approximately 2.''
\end{quote}

Note the careful phrasing: ``appears to be'' (German: ``erscheint als''/``scheint zu sein''), not ``is.'' The original authors were precise: the spectral dimension is a parameter extracted from correlation functions, not the dimension of physical space.

However, subsequent literature---particularly reviews and popular accounts---began using abbreviated terminology:
\begin{itemize}
\item ``Dimension flow'' (replacing ``spectral dimension variation'')
\item ``Running dimension'' (by analogy with running coupling constants)
\item ``Spacetime is 2D at the Planck scale'' (popular simplification)
\end{itemize}

This terminological drift led to the conflation of:
\begin{enumerate}
\item The mathematical parameter $d_s(\tau)$ (spectral dimension)
\item The physical concept of ``dimension of space''
\item The effective number of dynamical degrees of freedom
\end{enumerate}

\subsubsection{The German vs. English Distinction}

Interestingly, German physics literature has maintained clearer distinctions:
\begin{itemize}
\item ``Spektrale Dimension'' (mathematical parameter)
\item ``Effektive Dimension'' (physics of accessible modes)
\item ``Raumdimension'' or ``Topologische Dimension'' (geometric dimension of space)
\end{itemize}

The compound nature of German allows for more precise modifiers. English (and Chinese translations) lost some of this precision when ``spectral dimension'' was abbreviated to ``dimension'' in casual usage.

\subsubsection{Chinese Terminology: Translation Challenges}

The Chinese translation ``谱维度'' (pǔ wéidù) compounds the ambiguity:
\begin{itemize}
\item ``谱'' (spectrum) correctly captures the eigenvalue/spectral origin
\item But ``维度'' strongly connotes geometric dimension in Chinese physics education
\end{itemize}

Alternative translations that might have preserved precision:
\begin{itemize}
\item ``谱指数'' (spectral exponent)---emphasizes it's a scaling exponent
\item ``谱参数'' (spectral parameter)---neutral, technical term
\item ``有效自由度数'' (effective degree-of-freedom number)---physical interpretation
\end{itemize}

\subsection{The Three-Level Conceptual Framework}
\label{subsec:three_level}

To resolve the terminological confusion, we establish a rigorous three-level framework:

\begin{definition}[Level 1: Topological Dimension $d_{\text{topo}}$]
The topological dimension is the intrinsic dimensionality of the spacetime manifold, defined as the number of independent coordinates required to specify a point. For the physical spacetime considered in this review:
\begin{equation}
d_{\text{topo}} = 4 \quad \text{(three spatial + one temporal)}
\end{equation}
The topological dimension is a fixed property of the manifold and does not change with energy scale, probe resolution, or any physical parameter.
\end{definition}

\begin{definition}[Level 2: Spectral Dimension $d_s(\tau)$]
The spectral dimension is a \textbf{mathematical parameter} defined through the heat kernel trace:
\begin{equation}
d_s(\tau) = -2 \frac{d \ln K(\tau)}{d \ln \tau}
\label{eq:spectral_dimension_def}
\end{equation}
where $K(\tau) = \text{Tr}\, e^{\tau \Delta}$ is the return probability of diffusion processes.

**Critical clarification}: $d_s(\tau)$ is a \textbf{measure}, \textbf{probe}, or \textbf{diagnostic tool}. It is not a dimension in the geometric sense. The terminology ``dimension'' here is historical, deriving from the fact that for simple spaces, $d_s$ equals the topological dimension. For complex systems, $d_s$ quantifies the \textbf{scaling behavior} of diffusion, not the geometry of space.
\end{definition}

\begin{definition}[Level 3: Effective Degrees of Freedom $n_{\text{dof}}(E)$]
The effective number of dynamical degrees of freedom at energy scale $E$ is the count of independent directions in which excitations can propagate with energy cost less than or comparable to $E$.

In physical terms, if we probe a system with energy $E$, only those dynamical modes with excitation gap $E_{\text{gap}} \lesssim E$ can be accessed. Modes with $E_{\text{gap}} \gg E$ are effectively ``frozen'' or ``constrained.''

The relationship between spectral dimension and effective degrees of freedom is:
\begin{equation}
n_{\text{dof}}(E) \approx d_s(\tau) \quad \text{when} \quad E \sim \hbar/\tau
\label{eq:dof_relation}
\end{equation}
This is an approximate equality that holds when energy gaps are well-defined.
\end{definition}

\subsection{The Core Phenomenon: Energy-Dependent Mode Constraint}
\label{subsec:core_phenomenon}

The phenomenon this review addresses---variously called ``spectral dimension flow,'' ``running dimension,'' or ``dimensional reduction'' in the literature---is more precisely described as:

\begin{center}
\textbf{Energy-Dependent Constraint on Dynamical Degrees of Freedom}
\end{center}

\textbf{Physical mechanism}: 
Consider a system with topological dimension $d_{\text{topo}}$. Each independent direction of motion is associated with characteristic excitation modes. If a direction has a large energy gap $E_{\text{gap}}$ (due to centrifugal forces, gravitational redshift, quantum discreteness, etc.), then for probe energies $E \ll E_{\text{gap}}$, that direction is dynamically ``frozen'':
\begin{itemize}
\item Motion in that direction requires more energy than available
\item Excitations in that direction are exponentially suppressed
\item The direction exists geometrically but does not participate in low-energy dynamics
\end{itemize}

The ``flow'' in ``spectral flow'' refers to the continuous change in the \textbf{count} of accessible degrees of freedom as energy varies---not to any deformation or change in the geometric dimension of space.

\subsection{Structure and Terminology of This Review}
\label{subsec:structure}

In this review, we adopt the following precise terminology:
\begin{itemize}
\item \textbf{Spectral flow}: The variation of the spectral dimension parameter $d_s(\tau)$ with scale
\item \textbf{Effective dimension}: The number of accessible degrees of freedom $n_{\text{dof}}(E)$
\item \textbf{Mode constraint/freezing}: The physical mechanism by which high-gap modes decouple
\item We avoid ``dimension flow'' as ambiguous; when used, it refers specifically to the parameter $d_s(\tau)$, not physical space
\item We avoid ``dimensional reduction'' in favor of ``degree-of-freedom constraint''
\end{itemize}

This review is organized as follows. Section \ref{sec:foundations} establishes the mathematical framework, carefully distinguishing the spectral dimension as a mathematical probe from physical dimensions. Section \ref{sec:mechanisms} analyzes the three physical systems---rotating fluids, black holes, and quantum spacetime---demonstrating how distinct physical mechanisms (centrifugal forces, gravitational redshift, quantum discreteness) all lead to mode constraint with universal scaling. Section \ref{sec:evidence} reviews experimental and numerical evidence, interpreting observations in terms of mode constraint rather than geometric dimensional change. Section \ref{sec:implications} discusses implications for black hole physics, quantum gravity, and effective field theory. Section \ref{sec:outlook} concludes with open questions.

