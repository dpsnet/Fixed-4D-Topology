\section{引言}
\label{sec:introduction}

\subsection{现代物理学中的维度问题}

维度的概念位于我们理解物理现实的核心。
The concept of dimension lies at the heart of our understanding of physical reality.

从广义相对论的四维时空到弦理论所需的十或十一维,时空的维度对物理系统的行为有着深刻的影响。
From the four-dimensional spacetime of general relativity to the ten or eleven dimensions required by string theory, the dimensionality of space and time has profound implications for the behavior of physical systems.

然而,在量子尺度上,维度问题变得复杂。
However, the question of dimension becomes problematic at the quantum scale.

在可与普朗克长度相比较的距离上 $\ell_P \approx 1.6 \times 10^{-35}$ 米,经典时空的平滑流形描述失效,量子涨落占主导地位。
At distances comparable to the Planck length $\ell_P \approx 1.6 \times 10^{-35}$ m, the smooth manifold description of classical spacetime breaks down, and quantum fluctuations dominate.

这导致了\emph{谱维度流}的概念,即时空的有效维度随观测能量尺度而变化。
This has led to the concept of \emph{spectral dimension flow}, where the effective dimensionality of spacetime varies with the energy scale of observation.

\subsection{历史发展}

谱维度流的研究有着跨越多种量子引力方法的丰富历史:
The study of spectral dimension flow has a rich history spanning multiple approaches to quantum gravity:

\begin{itemize}
\item \textbf{因果动力学三角化(CDT)}:蒙特卡洛模拟显示在短距离上 $d_s = 2$,在大尺度上流变为 $d_s = 4$。
\item \textbf{Causal Dynamical Triangulations (CDT)}: Monte Carlo simulations show $d_s = 2$ at short distances, flowing to $d_s = 4$ at large scales.

\item \textbf{渐进安全}:泛函重整化群研究发现具有 $d_s \approx 2$ 的非高斯固定点。
\item \textbf{Asymptotic Safety}: Functional renormalization group studies find a non-Gaussian fixed point with $d_s \approx 2$.

\item \textbf{圈量子引力}:量子几何在普朗克尺度上通常表现出 $d_s = 2$。
\item \textbf{Loop Quantum Gravity}: Quantum geometry generically exhibits $d_s = 2$ at the Planck scale.

\item \textbf{弦理论}:世界面公式暗示修改的有效维度。
\item \textbf{String Theory}: Worldsheet formulations suggest modified effective dimensions.
\end{itemize}

\subsection{统一框架}

在本综述中,我们提出了一个统一框架,用于理解从量子引力到实验室系统的所有尺度上的维度流。
In this review, we present a unified framework for understanding dimension flow across all scales, from quantum gravity to laboratory systems.

核心结果是维度流参数的普适公式:
The central result is the universal formula for the dimension flow parameter:

\begin{equation}
c_1(d,w) = \frac{1}{2^{d-2+w}}
\label{eq:c1_formula_chinese}
\end{equation}

其中 $d$ 是空间维度,$w$ 代表时间维度。
where $d$ is the spatial dimension and $w$ represents time dimensions.

这个公式源于信息论考虑,并通过实验数据、数值模拟和理论一致性得到验证。
This formula emerges from information-theoretic considerations and is validated by experimental data, numerical simulations, and theoretical consistency.

\subsection{本综述的结构}

本综述的组织结构如下:
This review is organized as follows:

\begin{itemize}
\item 第 \ref{sec:foundations} 节介绍理论基础。
\item Section \ref{sec:foundations} presents the theoretical foundations.

\item 第 \ref{sec:correspondence} 节讨论三系统对应关系。
\item Section \ref{sec:correspondence} discusses the three-system correspondence.

\item 第 \ref{sec:experiments} 节回顾实验验证。
\item Section \ref{sec:experiments} reviews experimental validations.

\item 第 \ref{sec:applications} 节探索物理应用。
\item Section \ref{sec:applications} explores physical applications.

\item 第 \ref{sec:outlook} 节讨论开放问题和未来方向。
\item Section \ref{sec:outlook} discusses open questions and future directions.
\end{itemize}
