% 第5章:应用 - 逐句对照
\section{第五章:应用 / Chapter 5: Applications}
\label{sec:applications}

\textbf{[中]} 维度流理论在多个物理领域有着广泛的应用前景。

\textbf{[En]} Dimension flow theory has broad application prospects in multiple physics domains.

\subsection{引力波传播 / Gravitational Wave Propagation}

\textbf{[中]} 维度流预言了频率依赖的引力波传播速度修正。

\textbf{[En]} Dimension flow predicts frequency-dependent corrections to gravitational wave propagation speed.

\textbf{[中]} 在 $d_s \neq 4$ 的时空中,引力波的色散关系被修改为:

\textbf{[En]} In spacetime with $d_s \neq 4$, the gravitational wave dispersion relation is modified to:

\begin{equation}
\omega^2 = c^2 k^2 \left(\frac{k}{k_0}\right)^{4-d_s}
\end{equation}

\textbf{[中]} 其中 $k_0$ 是特征动量标度。

\textbf{[En]} where $k_0$ is the characteristic momentum scale.

\textbf{[中]} 这导致不同频率的引力波到达时间存在差异。

\textbf{[En]} This leads to arrival time differences for gravitational waves of different frequencies.

\textbf{[中]} 对于LIGO/Virgo观测的并合事件,可以检验这一预言。

\textbf{[En]} This prediction can be tested with merger events observed by LIGO/Virgo.

\subsection{宇宙学 / Cosmology}

\textbf{[中]} 早期宇宙的维度演化可能影响宇宙微波背景(CMB)的功率谱。

\textbf{[En]} Dimension evolution in the early universe may affect the cosmic microwave background (CMB) power spectrum.

\textbf{[中]} 在宇宙早期(高能量密度),有效维度可能接近2。

\textbf{[En]} In the early universe (high energy density), the effective dimension may be close to 2.

\textbf{[中]} 随着宇宙膨胀冷却,维度逐渐演化到4。

\textbf{[En]} As the universe expands and cools, the dimension gradually evolves to 4.

\textbf{[中]} 维度流可能在小尺度上引入额外的功率,需要通过高精度CMB实验来检验。

\textbf{[En]} Dimension flow may introduce additional power at small scales, which needs to be tested through high-precision CMB experiments.

\subsection{凝聚态系统 / Condensed Matter Systems}

\textbf{[中]} 维度流的概念可以应用于新型量子材料的设计。

\textbf{[En]} The concept of dimension flow can be applied to the design of novel quantum materials.

\textbf{[中]} 通过在材料中引入适当的约束或相互作用,可以调控有效维度。

\textbf{[En]} By introducing appropriate constraints or interactions in materials, the effective dimension can be tuned.

\textbf{[中]} 从而设计出具有新颖物理性质的量子材料。

\textbf{[En]} Thus enabling the design of quantum materials with novel physical properties.
