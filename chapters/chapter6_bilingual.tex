% 第6章:结论 - 逐句对照
\section{第六章:结论 / Chapter 6: Conclusion}
\label{sec:conclusion}

\subsection{总结 / Summary}

\textbf{[中]} 本文建立了维度流的统一理论框架。

\textbf{[En]} This review establishes a unified theoretical framework for dimension flow.

\textbf{[中]} 并通过三个独立的实验和数值系统验证了普适公式 $c_1(d,w)=1/2^{d-2+w}$。

\textbf{[En]} And validates the universal formula $c_1(d,w)=1/2^{d-2+w}$ through three independent experimental and numerical systems.

\textbf{[中]} 我们的主要成就包括:

\textbf{[En]} Our main achievements include:

\textbf{[中]} (1)提出了描述维度流的普适数学公式;

\textbf{[En]} (1) Proposing a universal mathematical formula describing dimension flow;

\textbf{[中]} (2)建立了旋转系统、黑洞和量子引力之间的三系统对应关系;

\textbf{[En]} (2) Establishing a three-system correspondence between rotation systems, black holes, and quantum gravity;

\textbf{[中]} (3)从Cu$_2$O里德堡激子实验中提取了维度流参数;

\textbf{[En]} (3) Extracting the dimension flow parameter from Cu$_2$O Rydberg exciton experiments;

\textbf{[中]} (4)提供了维度流在引力波、宇宙学和凝聚态系统中的可检验预言。

\textbf{[En]} (4) Providing testable predictions of dimension flow in gravitational waves, cosmology, and condensed matter systems.

\subsection{未来方向 / Future Directions}

\textbf{[中]} 未来研究方向包括:

\textbf{[En]} Future research directions include:

\textbf{[中]} (1)完成史瓦西几何谱维度流的严格解析证明;

\textbf{[En]} (1) Completing rigorous analytical proof of spectral dimension flow in Schwarzschild geometry;

\textbf{[中]} (2)在LHC上寻找维度流的粒子物理信号;

\textbf{[En]} (2) Searching for particle physics signals of dimension flow at the LHC;

\textbf{[中]} (3)利用第三代引力波探测器检验传播预言;

\textbf{[En]} (3) Testing propagation predictions using third-generation gravitational wave detectors;

\textbf{[中]} (4)发展量子模拟平台直接观测维度流。

\textbf{[En]} (4) Developing quantum simulation platforms for direct observation of dimension flow.

\subsection{最终评述 / Final Remarks}

\textbf{[中]} 维度流范式为理解时空的基本结构提供了一个全新的视角。

\textbf{[En]} The dimension flow paradigm provides a new perspective for understanding the fundamental structure of spacetime.

\textbf{[中]} 从量子引力到实验室物理,维度流统一了我们对自然界不同尺度上的理解。

\textbf{[En]} From quantum gravity to laboratory physics, dimension flow unifies our understanding of nature at different scales.

\vspace{1cm}
\begin{center}
\rule{0.7\textwidth}{0.5pt}\\[0.5em]
\textit{\textbf{[中]} 从量子涨落到宇宙结构,维度流统一了我们对时空的理解。}\\[0.3em]
\textit{\textbf{[En]} From quantum fluctuations to cosmic structures, dimension flow unifies our understanding of spacetime.}\\[0.3em]
\rule{0.7\textwidth}{0.5pt}
\end{center}
