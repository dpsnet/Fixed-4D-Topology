% 第3章:三系统对应 - 逐句对照
\section{第三章:三系统对应 / Chapter 3: Three-System Correspondence}
\label{sec:correspondence}

\textbf{[中]} 我们发现维度流在三个看似不同的物理系统中表现出普适行为:旋转系统、黑洞系统和量子引力。

\textbf{[En]} We find that dimension flow exhibits universal behavior across three seemingly different physical systems: rotation systems, black hole systems, and quantum gravity.

\subsection{旋转系统(E-6)/ Rotation Systems (E-6)}

\textbf{[中]} 在强旋转极限下,离心约束导致有效维度从4降低到约2.5。

\textbf{[En]} In the strong rotation limit, centrifugal constraints reduce the effective dimension from 4 to approximately 2.5.

\textbf{[中]} 这可以通过分析旋转参考系中的约束动力学来理解。

\textbf{[En]} This can be understood by analyzing constrained dynamics in rotating reference frames.

\textbf{[中]} 对于旋转角速度为 $\Omega$ 的系统,有效度规包含离心项。

\textbf{[En]} For a system with rotation angular velocity $\Omega$, the effective metric includes centrifugal terms.

\textbf{[中]} 当 $\Omega r \to 1$ 时,系统表现出类似黑洞的维度约化行为。

\textbf{[En]} When $\Omega r \to 1$, the system exhibits dimension reduction behavior similar to black holes.

\subsection{黑洞系统 / Black Hole Systems}

\textbf{[中]} 史瓦西黑洞的近视界几何近似于林德勒空间,导致谱维度 $d_s=2$。

\textbf{[En]} The near-horizon geometry of Schwarzschild black hole approximates Rindler space, leading to spectral dimension $d_s=2$.

\textbf{[中]} 定义乌龟坐标 $r_* = r + r_s \ln|r/r_s - 1|$,其中 $r_s = 2GM$ 是史瓦西半径。

\textbf{[En]} Define tortoise coordinate $r_* = r + r_s \ln|r/r_s - 1|$, where $r_s = 2GM$ is the Schwarzschild radius.

\textbf{[中]} 在 $r \to r_s$ 极限下,度规变为2维林德勒空间与2维球面的乘积。

\textbf{[En]} In the $r \to r_s$ limit, the metric becomes a product of 2D Rindler space and 2D sphere.

\textbf{[中]} 这是一个2维林德勒空间与2维球面的乘积,因此谱维度趋近于2。

\textbf{[En]} This is a product of 2D Rindler space and 2D sphere, so the spectral dimension approaches 2.

\subsection{量子引力 / Quantum Gravity}

\textbf{[中]} 因果动力学三角化(CDT)、渐进安全引力(ASG)和圈量子引力(LQG)的数值模拟都显示短距离维度降低到2。

\textbf{[En]} Numerical simulations in Causal Dynamical Triangulations (CDT), Asymptotic Safety Gravity (ASG), and Loop Quantum Gravity (LQG) all show dimension reduction to 2 at short distances.

\textbf{[中]} 在CDT模拟中,谱维度从紫外的 $d_s \approx 2$ 平滑过渡到大扩散时间的 $d_s \approx 4$。

\textbf{[En]} In CDT simulations, the spectral dimension smoothly transitions from $d_s \approx 2$ in the UV to $d_s \approx 4$ at large diffusion times.

\textbf{[中]} 过渡的特征时间尺度与普朗克时间相关。

\textbf{[En]} The characteristic time scale of the transition is related to the Planck time.

\textbf{[中]} 泛函重整化群方法预测维度流遵循动量标度的幂律行为。

\textbf{[En]} Functional renormalization group methods predict that dimension flow follows power-law behavior in momentum scale.

\subsection{三系统的统一描述 / Unified Description of Three Systems}

\textbf{[中]} 所有三个系统都遵循相同的普适行为:

\textbf{[En]} All three systems follow the same universal behavior:

\begin{equation}
d_{eff}(\varepsilon) = d_{min} + \frac{d_{max} - d_{min}}{1 + (\varepsilon/\varepsilon_c)^{c_1}}
\end{equation}

\textbf{[中]} 其中 $c_1$ 由系统的空间维度 $d$ 和时间维度 $w$ 通过公式 $c_1 = 1/2^{d-2+w}$ 确定。

\textbf{[En]} where $c_1$ is determined by the spatial dimension $d$ and time dimension $w$ of the system through the formula $c_1 = 1/2^{d-2+w}$.
