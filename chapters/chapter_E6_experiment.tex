% ============================================================================
% E-6 Experiment: Classical Tabletop Demonstration of Mode Constraint
% E-6实验:经典Tabletop模式约束演示
% ============================================================================

\section{The E-6 Experiment: A Classical Tabletop Demonstration}
\label{sec:E6_experiment}

The E-6 experiment provides a \textbf{classical tabletop demonstration} of the mode constraint phenomenon, showing that spectral dimension flow is not exclusive to quantum gravity but emerges from \textbf{energy-dependent constraints on dynamical degrees of freedom} in any physical system. This section details the experimental design, theoretical basis, and expected results.

% ----------------------------------------------------------------------------
\subsection{Conceptual Foundation}
\label{subsec:E6_concept}

\subsubsection{Core Insight: Classical Mode Constraint}

The E-6 experiment demonstrates that the phenomenon of "spectral dimension flow"—traditionally considered a quantum gravity effect—can be realized in \textbf{classical mechanical systems}. The key insight is:

\begin{equation}
\boxed{\text{Energy constraint} \Rightarrow \text{Mode freezing} \Rightarrow \text{Effective dimension reduction}}
\end{equation}

In quantum gravity, quantum fluctuations provide the energy-dependent constraint. In the E-6 experiment, \textbf{centrifugal forces} provide an analogous constraint mechanism.

\subsubsection{Correspondence Principle}

\begin{table}[htbp]
\centering
\caption{Correspondence: Quantum Gravity vs. E-6 Classical System}
\label{tab:E6_correspondence}
\begin{tabular}{@{}lcc@{}}
\toprule
\textbf{Feature} & \textbf{Quantum Gravity} & \textbf{E-6 Experiment} \\
\midrule
Driving mechanism & Quantum fluctuations & Centrifugal force \\
Energy scale & Planck energy $E_{\text{Pl}}$ & Rotational energy $E_{\text{rot}}$ \\
Constraint type & Quantum geometric & Classical mechanical \\
Dimension change & $4 \to 3 \to 2$ & $4 \to 3 \to 2$ \\
$c_1$ parameter & $0.125$ (quantum, $w=1$) & $0.25$ (classical, $w=0$) \\
\bottomrule
\end{tabular}
\end{table}

\subsubsection{Dimension Flow in the E-6 System}

The experiment uses a rotating system of small metal balls tethered by strings to a central rotating axis in a \textbf{microgravity environment}. As rotation speed increases:

\begin{itemize}
\item \textbf{Low energy} ($\omega \approx 0$): Balls float freely in 3D space
  \begin{equation}
  d_{\text{eff}} \approx 4 \quad (3\text{ space} + 1\text{ time})
  \end{equation}
  
\item \textbf{Medium energy} ($\omega \sim \omega_c$): Centrifugal forces constrain motion to 2D planes
  \begin{equation}
  d_{\text{eff}} \approx 3 \quad (2\text{ space} + 1\text{ time})
  \end{equation}
  
\item \textbf{High energy} ($\omega \gg \omega_c$): Strong constraint confines to 1D rings
  \begin{equation}
  d_{\text{eff}} \approx 2 \quad (1\text{ space} + 1\text{ time})
  \end{equation}
\end{itemize}

This directly mirrors the spectral dimension flow $d_s: 4 \to 3 \to 2$ predicted in quantum gravity.

% ----------------------------------------------------------------------------
\subsection{Experimental Design}
\label{subsec:E6_design}

\subsubsection{Apparatus}

\begin{table}[htbp]
\centering
\caption{E-6 Experimental Components}
\label{tab:E6_apparatus}
\begin{tabular}{@{}llcl@{}}
\toprule
\textbf{Component} & \textbf{Specification} & \textbf{Quantity} & \textbf{Purpose} \\
\midrule
Rotating axis & Diameter 10mm, length 50cm & 1 & Provide rotation \\
Stepper motor & 0-1000 rpm, precision control & 1 & Drive rotation \\
Strings & Length 10-25cm, nylon & 4 & Connect balls to axis \\
Metal balls & 1g, 5g, 10g, 20g masses & 4 & Test particles \\
Damping system & Adjustable 0-10 Ns/m & 4 & Simulate quantum fluctuations \\
Random signal generator & 0-1kHz & 1 & Control damping \\
High-speed cameras & 300fps, 1920$\times$1080 & 2 & Position recording \\
Position sensors & Laser, 0.1mm precision & 3 & 3D position measurement \\
Vibration system & 0-500Hz, 1mm amplitude & 1 & Random perturbations \\
\bottomrule
\end{tabular}
\end{table}

\subsubsection{Environment Requirements}

The experiment requires a \textbf{microgravity environment} to eliminate gravitational effects:
\begin{itemize}
\item Space laboratory (ISS or similar)
\item Drop tower (e.g., Bremen Drop Tower)
\item Parabolic flight aircraft
\item High-quality vacuum chamber to minimize air resistance
\end{itemize}

Temperature control: $25 \pm 2$°C to minimize thermal fluctuations.

\subsubsection{Multi-Mass Design}

Different mass balls probe different "coupling strengths" to the rotating field:

\begin{equation}
F_{\text{cf}} = m\omega^2 r \Rightarrow \text{heavier balls experience stronger constraint at same } \omega
\end{equation}

This allows testing the \textbf{mass-dependent mode constraint} predicted by the unified formula.

% ----------------------------------------------------------------------------
\subsection{Dimension Measurement Methods}
\label{subsec:E6_measurement}

\subsubsection{Box-Counting Method}

The primary method for measuring effective dimension:

\begin{enumerate}
\item Divide 3D space into cubes of size $\epsilon$
\item Count the number of cubes $N(\epsilon)$ containing at least one ball
\item Compute effective dimension:
  \begin{equation}
  d_{\text{eff}} = \lim_{\epsilon \to 0} \frac{\ln N(\epsilon)}{\ln(1/\epsilon)}
  \end{equation}
\end{enumerate}

\subsubsection{Angular Distribution Method}

Measure the deviation angle $\theta$ from the equatorial plane:

\begin{equation}
\theta = \arccos\left(\frac{z - z_0}{r}\right)
\end{equation}

The standard deviation $\sigma_\theta$ relates to effective dimension:

\begin{equation}
d_{\text{eff}} = 2 + \exp(-\sigma_\theta^2 / \sigma_0^2)
\end{equation}

where $\sigma_0$ is a calibration constant.

\subsubsection{Statistical Analysis}

For each rotation speed $\omega$:
\begin{itemize}
\item Record at least 1000 position measurements over 10 seconds
\item Compute $d_{\text{eff}}$ using both methods
\item Average over multiple runs to reduce statistical error
\item Estimate uncertainty: $\delta d_{\text{eff}} \approx 0.05$
\end{itemize}

% ----------------------------------------------------------------------------
\subsection{Experimental Protocol}
\label{subsec:E6_protocol}

\subsubsection{Four-Level Experimental Structure}

\textbf{Level 1: Basic Dimension Flow Verification}
\begin{itemize}
\item Rotation speeds: 0, 100, 200, ..., 1000 rpm
\item Measure $d_{\text{eff}}$ vs. $\omega$
\item Verify monotonic decrease $d_{\text{eff}}: 3.0 \to 2.2$
\item Expected transition region: 400-600 rpm
\end{itemize}

\textbf{Level 2: Mass-Dependent Constraint}
\begin{itemize}
\item Fixed speed: 500 rpm
\item Compare all four masses simultaneously
\item Verify: heavier balls $\Rightarrow$ lower $d_{\text{eff}}$
\item Test mass-dimension relation from unified formula
\end{itemize}

\textbf{Level 3: Quantum Fluctuation Analog}
\begin{itemize}
\item Activate random damping system
\item Vary damping strength: weak, medium, strong
\item Measure dimension fluctuations $\Delta d_{\text{eff}}$
\item Verify: stronger damping $\Rightarrow$ larger fluctuations
\item Demonstrate correspondence to quantum uncertainty
\end{itemize}

\textbf{Level 4: Fractal Structure Detection}
\begin{itemize}
\item High-resolution position tracking
\item Compute fractal dimension at multiple scales
\item Test for self-similarity in ball distribution
\item Search for log-periodic oscillations
\end{itemize}

% ----------------------------------------------------------------------------
\subsection{Theoretical Predictions}
\label{subsec:E6_predictions}

\subsubsection{Dimension-Energy Relation}

Based on the unified mode constraint formula, the expected behavior is:

\begin{equation}
d_{\text{eff}}(E) = d_{\text{IR}} - \frac{d_{\text{IR}} - d_{\text{UV}}}{1 + e^{(E - E_c)/(c_1 E_c)}}
\end{equation}

where:
\begin{itemize}
\item $d_{\text{IR}} = 4$ (4D spacetime at low energy)
\item $d_{\text{UV}} = 2$ (2D limit at high energy)
\item $c_1 = 0.25$ (classical value for $w=0$)
\item $E_c \sim \frac{1}{2}m\omega_c^2 r^2$ (critical rotational energy)
\end{itemize}

\subsubsection{Expected Results}

\begin{table}[htbp]
\centering
\caption{Expected Dimension Values at Different Rotation Speeds}
\label{tab:E6_expected}
\begin{tabular}{@{}ccc@{}}
\toprule
\textbf{Rotation Speed (rpm)} & \textbf{$E_{\text{rot}}$ (relative)} & \textbf{Expected $d_{\text{eff}}$} \\
\midrule
0 (stationary) & 0 & $3.0 \pm 0.1$ \\
200 & 0.04 & $2.9 \pm 0.1$ \\
400 & 0.16 & $2.7 \pm 0.1$ \\
600 & 0.36 & $2.5 \pm 0.1$ \\
800 & 0.64 & $2.3 \pm 0.1$ \\
1000 & 1.00 & $2.2 \pm 0.1$ \\
\bottomrule
\end{tabular}
\end{table}

\subsubsection{Mass-Dependence Prediction}

At fixed $\omega = 500$ rpm:

\begin{equation}
d_{\text{eff}}(m) = d_{\text{eff}}^{(0)} - \alpha \ln(m/m_0)
\end{equation}

where $\alpha \approx 0.1-0.2$ is determined by the constraint geometry.

Expected values:
\begin{itemize}
\item 1g ball: $d_{\text{eff}} \approx 2.7$
\item 20g ball: $d_{\text{eff}} \approx 2.3$
\end{itemize}

% ----------------------------------------------------------------------------
\subsection{Connection to the Unified Framework}
\label{subsec:E6_connection}

\subsubsection{Why $c_1 = 0.25$ for Classical Systems}

The E-6 experiment represents a \textbf{classical constraint} ($w=0$) in 4D space. According to the unified formula:

\begin{equation}
c_1(4, 0) = \frac{1}{2^{4-2+0}} = \frac{1}{4} = 0.25
\end{equation}

This differs from quantum gravity systems where $w=1$ gives $c_1 = 0.125$.

The larger $c_1$ in classical systems reflects:
\begin{enumerate}
\item \textbf{Deterministic constraints}: Classical centrifugal forces create sharp boundaries
\item \textbf{No tunneling}: Unlike quantum systems, classical particles cannot tunnel through barriers
\item \textbf{Coherent motion}: All particles respond identically to the constraint
\end{enumerate}

\subsubsection{Universality Verification}

The E-6 experiment tests the universal aspects of mode constraint:

\begin{enumerate}
\item \textbf{Cross-scale validity}: Same formula works from Planck scale to tabletop
\item \textbf{Cross-domain validity}: Same physics in quantum and classical regimes
\item \textbf{Mechanism independence}: Centrifugal force $
eq$ quantum fluctuations, but same outcome
\end{enumerate}

% ----------------------------------------------------------------------------
\subsection{Significance and Implications}
\label{subsec:E6_significance}

\subsubsection{For Quantum Gravity Research}

The E-6 experiment provides:
\begin{itemize}
\item An \textbf{analogue system} for studying quantum gravity effects
\item A \textbf{testing ground} for theoretical predictions
\item An \textbf{intuitive model} for understanding dimension flow
\item Evidence that dimension flow is \textbf{not necessarily quantum}
\end{itemize}

\subsubsection{For Fundamental Physics}

Key insights from the E-6 experiment:
\begin{enumerate}
\item \textbf{Dimension is dynamical}: Not fixed, but energy-dependent
\item \textbf{Constraints reduce dimension}: Any strong constraint freezes modes
\item \textbf{Universality}: The $c_1$ formula applies across all systems
\item \textbf{Emergent spacetime}: Dimension emerges from dynamics, not fundamental
\end{enumerate}

\subsubsection{Pedagogical Value}

The E-6 experiment makes quantum gravity concepts accessible:
\begin{itemize}
\item Visual demonstration of "dimension flow"
\item Hands-on experience with mode constraint
\item Intuitive understanding of why dimension changes
\item Direct connection between energy and geometry
\end{itemize}

% ----------------------------------------------------------------------------
\subsection{Status and Future Directions}
\label{subsec:E6_status}

\subsubsection{Current Status}

The E-6 experiment is currently a \textbf{conceptual design} awaiting:
\begin{itemize}
\item Microgravity facility access
\item Funding for apparatus construction
\item Collaboration with space agencies or drop tower facilities
\end{itemize}

\subsubsection{Proposed Variants}

\textbf{Ground-based version}: Using magnetic levitation to approximate microgravity

\textbf{Fluid dynamics version}: Using rotating fluid to visualize dimension flow

\textbf{Optical analogue}: Using light propagation in rotating media

\subsubsection{Integration with Other Tests}

The E-6 experiment complements other mode constraint tests:
\begin{itemize}
\item \textbf{Cu$_2$O excitons}: Quantum condensed matter test
\item \textbf{Hyperbolic manifolds}: Mathematical/numerical test
\item \textbf{CDT simulations}: Quantum gravity test
\item \textbf{E-6 experiment}: Classical mechanical test
\end{itemize}

Together, these tests span the full range of physical systems predicted to exhibit mode constraint.
