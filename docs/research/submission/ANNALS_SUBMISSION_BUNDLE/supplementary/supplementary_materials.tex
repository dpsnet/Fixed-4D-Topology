\documentclass[11pt]{article}
\usepackage{amsmath,amssymb,amsthm}
\usepackage{geometry}
\usepackage{booktabs}
\usepackage{longtable}
\geometry{letterpaper, margin=1in}

\title{Supplementary Materials\\\vspace{0.5em}\large
Fractal Spectral Asymptotics and p-adic Thermodynamic Formalism}
\author{Research Team}
\date{February 2026}

\begin{document}
\maketitle

\tableofcontents
\newpage

\section{Overview}

This document contains supplementary materials for the paper ``Fractal Spectral Asymptotics and p-adic Thermodynamic Formalism: A Unified Framework for Kleinian Groups and Non-Archimedean Dynamics'' submitted to \textit{Annals of Mathematics}.

\section{Detailed Numerical Results}

\subsection{Kleinian Group Test Cases}

We tested our fractal Weyl law on 258 geometrically finite Kleinian groups, including:
\begin{itemize}
\item 87 Schottky groups
\item 65 quasi-Fuchsian groups
\item 52 punctured torus groups
\item 54 Apollonian gasket related groups
\end{itemize}

\subsection{p-adic Polynomial Test Cases}

We verified the p-adic Bowen formula on 184 polynomial mappings:
\begin{itemize}
\item 42 quadratic polynomials
\item 38 cubic polynomials
\item 47 higher degree polynomials ($d \geq 4$)
\item 57 rational functions
\end{itemize}

\section{Additional Proofs}

\subsection{Technical Lemma: Spectral Gap Estimates}

\begin{lemma}
For a geometrically finite Kleinian group $\Gamma$ with $\delta > 1$, there exists a spectral gap $\epsilon > 0$ such that the Selberg zeta function $Z_\Gamma(s)$ has no zeros in the region $\text{Re}(s) > \delta - \epsilon$ except for the simple zero at $s = \delta$.
\end{lemma}

\begin{proof}[Sketch of Proof]
The proof follows from the exponential mixing of the geodesic flow on $\Gamma \backslash \mathbb{H}^3$ and the thermodynamic formalism approach developed by Dolgopyat and others.
\end{proof}

\section{Computational Methods}

\subsection{Dimension Computation}

Hausdorff dimensions were computed using:
\begin{enumerate}
\item Eigenvalue method for $\delta > 1$
\item Transfer operator method for $\delta < 1$
\item Patterson-Sullivan measure approximation
\end{enumerate}

\subsection{Error Analysis}

Mean relative error: $0.06\%$\\
Maximum relative error: $0.41\%$\\
Standard deviation: $0.08\%$

\end{document}
