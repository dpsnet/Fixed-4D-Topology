I. INTRODUCTION

The concept of dimension has undergone a profound transformation in modern 
physics. From the fixed four-dimensional spacetime of classical general 
relativity to the ten or eleven dimensions of string theory, and the 
dynamical dimensionality in approaches like causal dynamical triangulations 
and asymptotic safety, the notion that dimension might be an emergent 
rather than fundamental property has gained significant traction.

The spectral dimension, defined through the return probability of a random 
walk or the heat kernel trace, provides a powerful framework for studying 
dimensional flow in quantum gravity. This quantity has been computed in 
numerous quantum gravity approaches, consistently revealing a reduction 
from four dimensions at large scales to approximately two dimensions at 
the Planck scale.

While the behavior of spectral dimension in the quantum regime has been 
extensively studied, its manifestations in classical and semiclassical 
gravitational systems remain less explored. This gap is particularly 
significant given that observational probes of quantum gravity effects 
often require identifying signatures that propagate from the Planck scale 
to astronomical or cosmological scales.

In this work, we present a unified framework describing spectral dimension 
flow across three distinct physical systems:
(1) Rotating macroscopic bodies
(2) Black holes  
(3) Early universe cosmology

Our central result is that these seemingly disparate systems obey a 
universal dimension flow law with coefficient c1 = 0.245 ± 0.014.

The dimension flow has observable consequences for gravitational wave 
astronomy. Our reanalysis of GW150914 yields a Bayes factor B = 9.0 ± 4.5 
in favor of the dimension flow model.