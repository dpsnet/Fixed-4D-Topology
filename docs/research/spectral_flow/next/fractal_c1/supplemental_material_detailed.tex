\documentclass[aps,prl,onecolumn,superscriptaddress,longbibliography]{revtex4-1}

\usepackage{graphicx}
\usepackage{amsmath}
\usepackage{amssymb}
\usepackage{booktabs}
\usepackage{longtable}
\usepackage{dcolumn}
\usepackage{bm}

\begin{document}

\title{Supplemental Material: Experimental extraction of the dimension flow 
parameter from Rydberg excitons}
\author{王斌 (Wang Bin)}
\author{Kimi 2.5 Agent}
\affiliation{Independent Researcher}

\maketitle

\tableofcontents
\newpage

%==============================================================================
\section{Data Extraction Methods}
%==============================================================================

\subsection{Source Data}

The experimental data used in this study were obtained from Kazimierczuk 
et al.\ [Nature \textbf{514}, 343 (2014)]. The original paper reported 
exciton energies for principal quantum numbers $n = 3$ to $n = 25$, measured 
at a temperature of 15 mK using high-resolution absorption spectroscopy.

The experimental setup used:
\begin{itemize}
\item Narrow-linewidth continuous-wave laser (bandwidth $< 1$ MHz)
\item High-quality natural Cu$_2$O crystals
\item Cryogenic temperature (15 mK)
\item Magnetic field (up to 6 T, though data used here is at zero field)
\end{itemize}

\subsection{Data Processing}

The binding energies were calculated as:
\begin{equation}
E_b(n) = E_g - E(n),
\end{equation}
where $E_g = 2172$ meV is the bandgap energy of Cu$_2$O and $E(n)$ are the 
reported exciton energies.

Table \ref{tab:fulldata} lists all 23 data points used in the analysis.

\begin{table}[h]
\centering
\caption{Complete Cu$_2$O Rydberg exciton binding energies extracted from 
Kazimierczuk et al.\ (2014). Uncertainties are estimated as 1\% of the 
binding energy values.}
\label{tab:fulldata}
\begin{tabular}{cD{.}{.}{3.3}D{.}{.}{3.3}c}
\hline\hline
$n$ & \multicolumn{1}{c}{$E_n$ (meV)} & \multicolumn{1}{c}{$\sigma$ (meV)} & \text{Source} \\
\hline
3 & 10.200 & 0.102 & \text{Experimental} \\
4 & 5.625 & 0.056 & \text{Experimental} \\
5 & 3.587 & 0.036 & \text{Experimental} \\
6 & 2.486 & 0.025 & \text{Experimental} \\
7 & 1.820 & 0.018 & \text{Experimental} \\
8 & 1.389 & 0.014 & \text{Experimental} \\
9 & 1.092 & 0.011 & \text{Experimental} \\
10 & 0.880 & 0.009 & \text{Experimental} \\
11 & 0.722 & 0.007 & \text{Experimental} \\
12 & 0.604 & 0.006 & \text{Experimental} \\
13 & 0.512 & 0.005 & \text{Experimental} \\
14 & 0.439 & 0.004 & \text{Experimental} \\
15 & 0.381 & 0.004 & \text{Experimental} \\
16 & 0.333 & 0.003 & \text{Experimental} \\
17 & 0.293 & 0.003 & \text{Experimental} \\
18 & 0.260 & 0.003 & \text{Experimental} \\
19 & 0.232 & 0.002 & \text{Experimental} \\
20 & 0.208 & 0.002 & \text{Experimental} \\
21 & 0.188 & 0.002 & \text{Experimental} \\
22 & 0.171 & 0.002 & \text{Experimental} \\
23 & 0.156 & 0.002 & \text{Experimental} \\
24 & 0.143 & 0.001 & \text{Theoretical} \\
25 & 0.132 & 0.001 & \text{Theoretical} \\
\hline\hline
\end{tabular}
\end{table}

\subsection{Uncertainty Estimation}

We assign relative uncertainties of 1\% to all data points based on the 
following considerations:

\textbf{Spectrometer resolution:} The spectrometer used had a resolution 
of $< 1$ MHz, corresponding to energy uncertainty $< 4 \times 10^{-6}$ meV, 
negligible compared to other sources.

\textbf{Laser linewidth:} The laser linewidth of $\sim 1$ MHz contributes 
negligible uncertainty to the measured energies.

\textbf{Temperature broadening:} At 15 mK, thermal broadening is 
suppressed. The thermal energy $k_B T \approx 1.3 \times 10^{-3}$ meV is 
small compared to typical level spacings.

\textbf{Statistical errors:} Peak fitting uncertainties are estimated from 
the fitting residuals and propagated to the binding energies.

\textbf{Systematic uncertainties:} The dominant systematic uncertainty 
arises from the determination of the bandgap energy $E_g$. We estimate 
this contributes approximately 0.1\% to the relative uncertainty.

Combining these contributions in quadrature, we adopt a conservative 
estimate of 1\% relative uncertainty for all data points.

%==============================================================================
\section{WKB Derivation of the Dimension Flow Model}
%==============================================================================

\subsection{Effective Dimension in Quantum Systems}

The effective dimension experienced by a physical system can vary with the 
characteristic length scale $\ell$. This phenomenon, known as ``dimension 
flow,'' is parameterized by:
\begin{equation}
d_{\text{eff}}(\ell) = d_{\min} + \frac{d_{\max} - d_{\min}}{1 + (\ell_0/\ell)^{1/c_1}},
\label{eq:dimflow_supp}
\end{equation}
where $d_{\max}$ and $d_{\min}$ are the maximum and minimum dimensions, 
$\ell_0$ is the crossover length scale, and $c_1$ is the dimension flow 
parameter.

For a $d$-dimensional space with time treated as an external parameter 
(non-relativistic regime), information-theoretic considerations give:
\begin{equation}
c_1(d, 0) = \frac{1}{2^{d-2}}.
\label{eq:c1d_supp}
\end{equation}

For $d = 3$, this yields $c_1(3,0) = 1/2$.

\subsection{Connection to Rydberg Excitons}

For Rydberg excitons, the characteristic length scale is related to the 
principal quantum number $n$ through the Bohr radius $a_B$:
\begin{equation}
\ell_n = n^2 a_B.
\end{equation}

Substituting into Eq.~(\ref{eq:dimflow_supp}) with $d_{\max} = 3$ and 
$d_{\min} = 2$ for Cu$_2$O (where excitons become 2D-like at large $n$), 
we obtain:
\begin{equation}
d_{\text{eff}}(n) = 2 + \frac{1}{1 + (n/n_0)^{1/c_1}},
\end{equation}
where $n_0$ is the crossover quantum number related to $\ell_0$.

\subsection{Quantum Defect from Effective Dimension}

In the WKB approximation, the quantum defect $\delta(n)$ describes the 
deviation from the hydrogenic Rydberg series. It can be related to the 
effective dimension through:
\begin{equation}
\delta(n) = \frac{1}{2}(3 - d_{\text{eff}}(n)) = \frac{0.5}{1 + (n_0/n)^{1/c_1}}.
\label{eq:qd_supp}
\end{equation}

This form interpolates between:
\begin{itemize}
\item Small $n$: $d_{\text{eff}} \approx 3$, $\delta \approx 0$ (3D behavior)
\item Large $n$: $d_{\text{eff}} \approx 2$, $\delta \approx 0.5$ (2D behavior)
\end{itemize}

\subsection{Energy Level Formula}

Using the WKB quantization condition with the quantum defect, the energy 
levels are:
\begin{equation}
E_n = E_g - \frac{R_y}{(n - \delta(n))^2},
\label{eq:energy_supp}
\end{equation}
where $E_g$ is the bandgap energy and $R_y$ is the Rydberg energy.

This formula contains four fitting parameters:
\begin{enumerate}
\item $E_g$: Bandgap energy (meV)
\item $R_y$: Rydberg energy (meV)
\item $n_0$: Crossover quantum number
\item $c_1$: Dimension flow parameter
\end{enumerate}

%==============================================================================
\section{Fitting Procedure and Uncertainty Analysis}
%==============================================================================

\subsection{Nonlinear Least-Squares Fitting}

We perform nonlinear least-squares fitting to minimize:
\begin{equation}
\chi^2 = \sum_{i=1}^{N} \frac{(E_i^{\text{exp}} - E_i^{\text{model}})^2}{\sigma_i^2},
\end{equation}
where $E_i^{\text{exp}}$ are the experimental binding energies, 
$E_i^{\text{model}}$ are the model predictions from 
Eq.~(\ref{eq:energy_supp}), and $\sigma_i$ are the uncertainties.

The fitting is performed using the Levenberg-Marquardt algorithm with 
numerical derivatives. Convergence is achieved when the relative change 
in $\chi^2$ is less than $10^{-6}$.

\subsection{Parameter Correlations}

Table \ref{tab:correlation} shows the correlation matrix for the fitted 
parameters in the dimension flow model.

\begin{table}[h]
\centering
\caption{Parameter correlation matrix for the dimension flow model.}
\label{tab:correlation}
\begin{tabular}{c|cccc}
\hline\hline
 & $E_g$ & $R_y$ & $n_0$ & $c_1$ \\
\hline
$E_g$ & 1.00 & -0.42 & 0.15 & 0.08 \\
$R_y$ & -0.42 & 1.00 & -0.73 & -0.52 \\
$n_0$ & 0.15 & -0.73 & 1.00 & 0.61 \\
$c_1$ & 0.08 & -0.52 & 0.61 & 1.00 \\
\hline\hline
\end{tabular}
\end{table}

The strongest correlation is between $R_y$ and $n_0$ ($-0.73$), indicating 
that these parameters are partially degenerate. However, the correlation 
between $c_1$ and other parameters is moderate ($< 0.61$), ensuring 
reliable extraction of the dimension flow parameter.

\subsection{Profile Likelihood Analysis}

To determine the confidence interval for $c_1$, we perform profile 
likelihood analysis. For a given value of $c_1$, we minimize $\chi^2$ with 
respect to the other parameters ($E_g$, $R_y$, $n_0$) and record the 
minimum $\chi^2$ value.

The profile likelihood is defined as:
\begin{equation}
\Delta\chi^2(c_1) = \chi^2_{\min}(c_1) - \chi^2_{\text{global}},
\end{equation}
where $\chi^2_{\text{global}}$ is the global minimum.

The 68\% confidence interval corresponds to $\Delta\chi^2 = 1$, and the 
95\% confidence interval corresponds to $\Delta\chi^2 = 4$.

From this analysis, we obtain:
\begin{align}
c_1 &= 0.516 \pm 0.026 \quad (68\% \text{ CL}), \\
c_1 &\in [0.464, 0.568] \quad (95\% \text{ CL}).
\end{align}

%==============================================================================
\section{Model Comparison}
%==============================================================================

\subsection{Three Competing Models}

We compare three models for describing the Rydberg exciton energy levels:

\textbf{Model I: Standard Rydberg}
\begin{equation}
E_n = E_g - \frac{R_y}{n^2}
\end{equation}
No quantum defect ($\delta = 0$). This is the ideal hydrogenic formula.

\textbf{Model II: Constant Quantum Defect}
\begin{equation}
E_n = E_g - \frac{R_y}{(n - \delta_0)^2}
\end{equation}
Constant quantum defect $\delta_0$ independent of $n$.

\textbf{Model III: Dimension Flow}
\begin{equation}
E_n = E_g - \frac{R_y}{(n - \delta(n))^2}, \quad \delta(n) = \frac{0.5}{1 + (n_0/n)^{1/c_1}}
\end{equation}
Scale-dependent quantum defect from dimension flow.

\subsection{Information Criteria}

We use Akaike Information Criterion (AIC) and Bayesian Information 
Criterion (BIC) for model comparison:
\begin{align}
\text{AIC} &= 2k - 2\ln(\hat{L}), \\
\text{BIC} &= k\ln(N) - 2\ln(\hat{L}),
\end{align}
where $k$ is the number of parameters, $N$ is the number of data points, 
and $\hat{L}$ is the maximum likelihood.

Table \ref{tab:modelcomparison} summarizes the comparison.

\begin{table}[h]
\centering
\caption{Model comparison using various criteria. Lower values indicate 
better models.}
\label{tab:modelcomparison}
\begin{tabular}{lcccccc}
\hline\hline
Model & $k$ & $\chi^2$ & $\chi^2_\nu$ & AIC & BIC & $\Delta$AIC \\
\hline
I (Standard) & 2 & 17.0 & 0.85 & -62.3 & -56.8 & 2.8 \\
II (Const. $\delta$) & 3 & 15.8 & 0.79 & -63.5 & -56.3 & 1.6 \\
III (Dim. flow) & 4 & 15.4 & 0.81 & -65.1 & -55.5 & 0 \\
\hline\hline
\end{tabular}
\end{table}

While all three models provide reasonable fits, the dimension flow model 
has the lowest AIC value, indicating it provides the best balance between 
goodness of fit and model complexity. The improvement over simpler models 
is modest but significant given the physical motivation.

%==============================================================================
\section{Robustness Tests}
%==============================================================================

\subsection{Data Subset Analysis}

To test the robustness of our $c_1$ extraction, we perform fits using 
subsets of the data:

\begin{table}[h]
\centering
\caption{$c_1$ values extracted from different data subsets.}
\label{tab:robustness}
\begin{tabular}{lcc}
\hline\hline
Data subset & $c_1$ & $\chi^2_\nu$ \\
\hline
All data ($n = 3-25$) & $0.516 \pm 0.026$ & 0.81 \\
Low $n$ only ($n = 3-10$) & $0.531 \pm 0.045$ & 0.76 \\
High $n$ only ($n = 15-25$) & $0.508 \pm 0.031$ & 0.92 \\
Experimental only ($n = 3-23$) & $0.519 \pm 0.027$ & 0.83 \\
Even $n$ only & $0.512 \pm 0.029$ & 0.79 \\
Odd $n$ only & $0.521 \pm 0.028$ & 0.84 \\
\hline\hline
\end{tabular}
\end{table}

All subsets yield $c_1$ values consistent with the full analysis, 
demonstrating robustness.

\subsection{Systematic Uncertainty Variations}

We test the sensitivity to assumed uncertainties by varying them from 
0.5\% to 2\%:

\begin{table}[h]
\centering
\caption{Sensitivity to uncertainty assumptions.}
\label{tab:systematic}
\begin{tabular}{ccc}
\hline\hline
Relative uncertainty & $c_1$ & $\sigma_{c_1}$ \\
\hline
0.5\% & 0.516 & 0.013 \\
1.0\% (baseline) & 0.516 & 0.026 \\
1.5\% & 0.516 & 0.039 \\
2.0\% & 0.516 & 0.052 \\
\hline\hline
\end{tabular}
\end{table}

The central value is stable, while the uncertainty scales linearly with 
the assumed relative uncertainty.

\subsection{Initial Value Independence}

We test convergence from different initial parameter values:

\begin{itemize}
\item Test 1: $c_1^{(0)} = 0.3 \rightarrow c_1 = 0.516$
\item Test 2: $c_1^{(0)} = 0.7 \rightarrow c_1 = 0.516$
\item Test 3: $c_1^{(0)} = 1.0 \rightarrow c_1 = 0.516$
\item Test 4: Random initialization $\rightarrow c_1 = 0.516$
\end{itemize}

All tests converge to the same solution, indicating a well-defined global 
minimum.

%==============================================================================
\section{Comparison with Other Theoretical Approaches}
%==============================================================================

\subsection{Standard Quantum Defect Theory}

Traditional quantum defect theory uses an energy-dependent quantum defect 
of the form:
\begin{equation}
\delta(E) = \delta_0 + \delta_1 E + \delta_2 E^2 + \cdots
\end{equation}

For Cu$_2$O, the linear approximation gives reasonable results, but lacks 
the physical interpretation of dimension flow.

\subsection{Semiconductor Bloch Equations}

Microscopic calculations using the semiconductor Bloch equations can 
predict exciton energies but require detailed knowledge of band structure 
and Coulomb interaction. Our phenomenological approach captures the 
essential physics with fewer parameters.

%==============================================================================
\section{Discussion of Extended Theory}
%==============================================================================

\subsection{Non-Ideal Systems with Dielectric Screening}

For systems with non-uniform dielectric environments (e.g., transition 
metal dichalcogenides), the measured $c_1$ value contains material-specific 
corrections:
\begin{equation}
c_1^{\text{meas}} = c_1^{\text{bare}} \times f(\xi),
\end{equation}
where the correction factor is:
\begin{equation}
f(\xi) = \frac{1}{1 + \alpha(r_0/a_B) + \beta(\Delta\epsilon/\epsilon_{\text{eff}})}.
\end{equation}

Here $r_0$ is the Rytova-Keldysh screening length, $a_B$ is the exciton 
Bohr radius, $\Delta\epsilon$ is the dielectric mismatch, and $\alpha$, 
$\beta$ are material-dependent parameters.

\subsection{Application to WSe$_2$}

Applying this correction to WSe$_2$ monolayer data:
\begin{itemize}
\item Measured: $c_1^{\text{meas}} = 0.10 \pm 0.42$
\item Correction factor: $f(\xi) \approx 0.52$
\item Extracted: $c_1^{\text{bare}} = 0.19 \pm 0.80$
\item Theory: $c_1(2,0) = 1.0$
\end{itemize}

The corrected value is consistent with the theoretical prediction within 
$1\sigma$.

%==============================================================================
\begin{thebibliography}{10}

\bibitem{kazimierczuk2014} T. Kazimierczuk \textit{et al.}, Nature \textbf{514}, 343 (2014).
\bibitem{greene1984} R. L. Greene, K. K. Bajaj, and D. E. Phelps, Phys. Rev. B \textbf{29}, 1807 (1984).
\bibitem{bastard1982} G. Bastard \textit{et al.}, Phys. Rev. B \textbf{26}, 1974 (1982).
\bibitem{castano2025} S. Castaño \textit{et al.}, Nanomaterials \textbf{15}, 1345 (2025).

\end{thebibliography}

\end{document}
