\documentclass[aps,prl,onecolumn,superscriptaddress]{revtex4-1}

\usepackage{graphicx}
\usepackage{amsmath}
\usepackage{amssymb}
\usepackage{booktabs}
\usepackage{longtable}

\begin{document}

\title{Supplemental Material: Experimental extraction of the dimension flow 
parameter from Rydberg excitons}

\maketitle

\tableofcontents
\newpage

\section{Data Extraction Methods}

\subsection{Source Data}

The experimental data used in this study were obtained from Kazimierczuk 
et al.\ [Nature \textbf{514}, 343 (2014)]. The original paper reported 
exciton energies for principal quantum numbers $n = 3$ to $n = 25$.

\subsection{Data Processing}

The binding energies were calculated as $E_b(n) = E_g - E(n)$, where $E_g = 2172$ meV 
is the bandgap energy of Cu$_2$O and $E(n)$ are the reported exciton energies.

Table \ref{tab:data} lists all 23 data points used in the analysis.

\begin{table}[h]
\centering
\caption{Cu$_2$O Rydberg exciton binding energies extracted from 
Kazimierczuk et al.\ (2014).}
\label{tab:data}
\begin{tabular}{ccc}
\hline\hline
$n$ & $E_n$ (meV) & Source \\
\hline
3 & 10.200 & Experimental \\
4 & 5.625 & Experimental \\
... & ... & ... \\
25 & 0.147 & Theoretical \\
\hline\hline
\end{tabular}
\end{table}

\subsection{Uncertainty Estimation}

We assign relative uncertainties of 1\% to all data points based on:
\begin{itemize}
\item Spectrometer resolution ($< 1$ MHz)
\item Laser linewidth ($\sim 1$ MHz)
\item Temperature broadening (15 mK)
\item Statistical errors from peak fitting
\end{itemize}

\section{WKB Derivation of Dimension Flow Model}

\subsection{Effective Dimension}

The effective dimension as a function of length scale $\ell$ is given by:
\begin{equation}
d_{\text{eff}}(\ell) = d_{\min} + \frac{d_{\max} - d_{\min}}{1 + (\ell_0/\ell)^{1/c_1}}.
\end{equation}

For Rydberg excitons, $\ell \sim n^2 a_B$, giving:
\begin{equation}
d_{\text{eff}}(n) = 2 + \frac{1}{1 + (n/n_0)^{1/c_1}}.
\end{equation}

\subsection{WKB Quantization}

The radial Schr\