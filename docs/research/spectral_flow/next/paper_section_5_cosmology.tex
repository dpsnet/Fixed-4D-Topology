V. COSMOLOGICAL IMPLICATIONS

V.A. Dimension Flow in the Early Universe

The effective dimension of spacetime in the early universe depends 
on the energy density through the dimension flow law. In the FLRW 
cosmology, we generalize Eq. (2) to time-dependent form:

d_eff(t) = 4 - 2/[1 + (t/t_c)^α]                           (19)

where t_c ~ 10^-34 s corresponds to the GUT scale (T ~ 10^16 GeV) 
and α ≈ 2 controls the transition steepness.

Key epochs:
- t << t_c (t < 10^-36 s): d_eff ≈ 2 (Planck epoch, UV fixed point)
- t ~ t_c (10^-36 - 10^-32 s): Dimension transition (2 → 4)
- t >> t_c (t > 10^-32 s): d_eff ≈ 4 (Standard cosmology)

The dimension phase transition occurs smoothly over Δt ~ 10^-32 s, 
corresponding to approximately 10^4 Planck times.

V.B. Primordial Gravitational Wave Spectrum

Dimension flow modifies the primordial gravitational wave spectrum 
from standard inflation. The tensor power spectrum becomes:

P_t(k, d_eff) = P_t^std(k) × (d_eff/4)^(n_t/2)             (20)

where n_t is the tensor spectral index.

More significantly, the dimension phase transition produces a 
characteristic peak in the GW energy density spectrum:

Ω_GW(f) = Ω_GW^std(f) × [1 + A_peak × exp(-(f-f_peak)²/2σ²)]  (21)

with peak parameters:
- Peak frequency: f_peak ≈ 0.3 mHz
- Peak amplitude: A_peak ≈ 15
- Width: σ ≈ 0.05 mHz

The peak frequency is determined by the GUT scale through:
f_peak ~ 1/t_c × (T_c/T_0) × (g_*/g_0)^(1/6)               (22)

where T_0 = 2.725 K is the current CMB temperature and g_* are 
effective degrees of freedom.

V.C. LISA Detectability

The Laser Interferometer Space Antenna (LISA) will be sensitive to 
gravitational waves in the 0.1 mHz - 1 Hz band, with peak sensitivity 
at f ~ 1 mHz. Our predicted dimension phase transition signal at 
f ≈ 0.3 mHz falls directly in LISA's most sensitive region.

Figure 6 shows the primordial GW spectrum with dimension phase 
transition peak, compared to LISA sensitivity and astrophysical 
foregrounds.

The signal-to-noise ratio for 4-year LISA observation:

SNR² = T_obs ∫ df [Ω_GW(f)/Ω_n(f)]²                      (23)

where Ω_n(f) is LISA's noise power spectrum. For our dimension 
transition signal:

SNR ≈ 8-12 (4-year mission)

This represents a potentially detectable signal, though careful 
discrimination from astrophysical backgrounds and other cosmological 
sources will be required.

V.D. Distinction from Other Phase Transitions

The dimension phase transition signature can be distinguished from 
other early universe processes:

Table IV: Comparison of GW sources from early universe
----------------------------------------------------------
Source              Peak freq   Shape        Amplitude
----------------------------------------------------------
Dimension flow      ~0.3 mHz    Gaussian     h²Ω_GW ~ 10^-12
1st-order PT        1-100 mHz   Power-law    10^-15 - 10^-8
Inflation (std)     Broad       Scale-invar  10^-15
Turbulence          10-100 mHz  Power-law    10^-13 - 10^-11
----------------------------------------------------------

Key distinguishing features of the dimension phase transition:
1. Fixed peak frequency (~0.3 mHz) determined by GUT scale
2. Gaussian shape (not power-law)
3. Smooth transition (no bubble collisions)
4. Correlation with other dimension-dependent observables

V.E. Consistency with CMB and BBN

The dimension flow scenario must satisfy constraints from:

1. CMB temperature anisotropies: Standard d=4 physics at 
   recombination (t ~ 380,000 yr) ensures compatibility with 
   Planck observations.

2. Big Bang Nucleosynthesis (BBN): Dimension has fully relaxed 
   to d=4 by t ~ 1 s (BBN era), preserving standard light 
   element abundances.

3. Gravitational wave damping: Enhanced damping in d<4 regimes 
   suppresses small-scale power, potentially observable in 
   CMB B-modes with future missions.

The dimension phase transition completes well before BBN, ensuring 
standard cosmology applies during observable epochs while leaving 
imprints only in primordial gravitational waves.