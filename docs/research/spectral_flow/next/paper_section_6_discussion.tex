VI. DISCUSSION

VI.A. Summary of Results

We have presented a unified framework for dimension flow in 
gravitational systems, encompassing rotating bodies, black holes, 
and quantum gravity. Our key findings include:

1. A universal dimension flow law with coefficient c₁ = 0.245 ± 0.014, 
   consistent with the theoretical prediction c₁ = 1/4.

2. Observable signatures in gravitational wave signals, with 
   GW150914 showing moderate evidence (B = 9.0) for dimension flow.

3. A predicted peak in the primordial gravitational wave spectrum 
   at f ≈ 0.3 mHz, potentially detectable by LISA.

VI.B. Theoretical Implications

The dimension flow framework suggests that dimension is an emergent 
property rather than a fundamental constant. The UV fixed point at 
d = 2 aligns with predictions from:
- Causal dynamical triangulations
- Asymptotic safety
- Hořava-Lifshitz gravity

VI.C. Observational Prospects

Near-term tests:
- Extended GW150914-like analysis with O3/O4 LIGO data
- Population-level tests for systematic biases
- Null tests with binary neutron stars

Future tests:
- LISA detection of primordial GW peak (2030s)
- CMB spectral distortions from early dimension flow
- Laboratory analog systems

VI.D. Limitations and Future Work

Current limitations include:
- Sample size for c₁ determination (need N ~ 64k for 5σ)
- Single GW event analysis (need O(10-100) events)
- Simplified dimension transition model

Future work will address:
- Full census analysis (212k manifolds)
- Population-level GW studies
- Connection to quantum gravity frameworks