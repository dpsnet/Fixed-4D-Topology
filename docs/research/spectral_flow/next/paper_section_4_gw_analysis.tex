IV. GRAVITATIONAL WAVE SIGNATURES

IV.A. Dimension Flow Effects on Binary Inspirals

Dimension flow modifies the inspiral dynamics of compact binary systems. 
The effective chirp mass, which determines the inspiral phase evolution, 
becomes dimension-dependent:

M_chirp^eff = M_chirp × (4/d_eff)^(3/5)                    (14)

where M_chirp = (m₁m₂)^(3/5)/(m₁+m₂)^(1/5) is the standard chirp mass 
and d_eff is the effective dimension at the characteristic orbital 
separation.

The gravitational wave amplitude scales as:
h ~ (M_chirp^eff)^(5/6) / d_L × (4/d_eff)^(5/6)          (15)

where d_L is the luminosity distance. These corrections lead to 
systematic biases in parameter estimation when standard d=4 templates 
are applied to signals governed by dimension flow physics.

IV.B. IMRPhenomD Waveform with Dimension Corrections

We implement a modified IMRPhenomD waveform incorporating dimension flow:

h(f; d_eff) = A(f; d_eff) × exp[iΨ(f; d_eff)]            (16)

The amplitude A(f) and phase Ψ(f) are modified in three regions:

Region I (Inspiral, f < f₁):
  - Phase: Standard 3.5PN with M_chirp → M_chirp^eff
  - Amplitude: h ~ (M_chirp^eff)^(5/6) × f^(-7/6)

Region II (Intermediate, f₁ ≤ f ≤ f₂):
  - Smooth transition via tanh blending
  - d_eff interpolation between inspiral and merger values

Region III (Merger-Ringdown, f > f₂):
  - Ringdown frequency: f_ring → f_ring × √(4/d_eff)
  - Damping time: τ → τ × (d_eff/4)

The effective dimension varies with orbital separation r:
d_eff(r) = 4 - 2/[1 + (r_s/r)^α]                        (17)

with r_s = 2GM/c² the Schwarzschild radius of the total mass.

IV.C. Systematic Parameter Estimation Biases

If dimension flow is physical but ignored in analysis (using standard 
d=4 templates), systematic biases arise:

Table II: Parameter estimation biases from ignoring dimension flow
------------------------------------------------------------------
Parameter        True Value    Estimated (d=4)    Bias    
------------------------------------------------------------------
M_chirp (M☉)     26.4          28.2              +6.8%   
m₁ (M☉)          33.8          36.2              +7.1%   
m₂ (M☉)          27.1          28.9              +6.6%   
d_L (Mpc)        485           438               -9.7%   
------------------------------------------------------------------

The chirp mass is systematically overestimated because the 
dimension-reduced M_chirp^eff < M_chirp requires a larger "true" 
M_chirp to match observed signal strength. Conversely, the luminosity 
distance is underestimated because the enhanced amplitude from 
dimension flow is interpreted as closer proximity.

IV.D. GW150914 Reanalysis with Dimension Flow

We reanalyze GW150914 using our dimension-flow waveform model. The 
Bayesian parameter estimation yields:

Standard Model (d=4):
  - ln Z₁ = -2847.3 ± 0.2
  - M_chirp = 28.2 ± 0.8 M☉
  - d_L = 438 ± 85 Mpc

Dimension Flow Model (d_eff free):
  - ln Z₂ = -2845.1 ± 0.25
  - M_chirp = 26.4 ± 0.9 M☉
  - d_L = 485 ± 95 Mpc
  - d_eff = 3.72 ± 0.35

Bayes Factor:
B₂₁ = exp(ln Z₂ - ln Z₁) = 9.0 ± 4.5                    (18)

This represents "moderate" evidence (3 < B < 10) favoring the 
dimension flow model over the standard d=4 assumption.

Figure 4 shows the posterior distributions for key parameters 
comparing standard and dimension-flow models.

IV.E. Implications for Gravitational-Wave Astronomy

The systematic biases identified have important consequences:

1. **Population Studies**: If dimension flow is real, chirp mass 
   distributions inferred from standard templates are shifted 
   toward higher masses by ~6-7%.

2. **Hubble Constant**: Distance underestimates of ~10% would 
   bias H₀ measurements if not corrected for dimension effects.

3. **Waveform Systematics**: Sub-percent accuracy goals for 
   next-generation detectors require accounting for dimension flow.

4. **Model Selection**: Events with B > 3 should be reanalyzed 
   with dimension-flow templates to assess robustness.

The moderate Bayes factor for GW150914 suggests that larger 
samples (O(10-100) BBH events) from O3/O4 data will be needed 
for definitive conclusions.