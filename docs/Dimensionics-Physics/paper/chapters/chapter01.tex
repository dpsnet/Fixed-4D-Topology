% Chapter 1: Introduction and Background
% Dimensionics-Physics: Introduction

\section{From Constant to Variable Dimension}\label{sec:intro-dimension}

\subsection{Dimension in Classical Physics}

In classical physics, spacetime dimension is viewed as a \emph{fixed background}. Newtonian mechanics assumes absolute space and time with 3 spatial and 1 temporal dimension. Special relativity extends this to a 4-dimensional Minkowski spacetime, while general relativity maintains the topological dimension fixed at 4, albeit with curvature.

This paradigm of ``fixed dimension'' achieved tremendous success in 20th century physics, from atomic physics to cosmology. Both the Standard Model and $\Lambda$CDM cosmology are founded on 4-dimensional spacetime.

\subsection{The Challenge of Quantum Gravity}

However, when attempting to unify quantum mechanics with gravity, the assumption of fixed dimension faces fundamental challenges. At the Planck scale ($l_{\text{Pl}} \sim 10^{-35}$ m), quantum fluctuations make spacetime structure violently fluctuate. Traditional 4-dimensional quantum field theory develops non-renormalizable divergences at this scale.

Multiple approaches to quantum gravity suggest that at very small scales, the ``effective dimension'' may be lower than 4:
\begin{itemize}
    \item \textbf{Loop Quantum Gravity (LQG):} Spin networks exhibit 2-dimensional characteristics at Planck scale \cite{Modesto2009}
    \item \textbf{Causal Dynamical Triangulation (CDT):} Numerical simulations show spectral dimension $d_s \approx 2$ at UV \cite{Ambjorn2012}
    \item \textbf{String Theory:} Compactification of 10 or 26 dimensions suggests dimension is dynamical
    \item \textbf{Asymptotically Safe Gravity:} Renormalization group flow suggests UV fixed point \cite{Reuter2019}
\end{itemize}

These results collectively point to a profound insight: \emph{dimension may be energy-dependent}.

\section{Spectral Dimension: From Geometry to Physics}\label{sec:spectral-dim}

\subsection{Mathematical Definition}

\textbf{Spectral dimension} is a geometric quantity with deep physical meaning. On a metric space $(M, g)$, consider the solution to the heat equation:
\begin{equation}
    \frac{\partial u}{\partial t} + \Delta u = 0
\end{equation}

The asymptotic behavior of the heat kernel trace:
\begin{equation}\label{eq:heat-kernel}
    Z(t) = \mathrm{Tr}(e^{-t\Delta}) \sim t^{-d_s/2}, \quad t \to \infty
\end{equation}
defines the \textbf{spectral dimension} $d_s$.

\textbf{Key Properties}:
\begin{itemize}
    \item For smooth manifolds: $d_s = d$ (topological dimension)
    \item For fractal spaces: $d_s$ can be fractional
    \item $d_s$ depends on the scale considered
\end{itemize}

\subsection{Physical Interpretation}

The physical meaning of spectral dimension is the \emph{effective dimension ``perceived'' by probe particles}:
\begin{itemize}
    \item \textbf{High-energy probes} (small $t$): Probe short-distance structure, may perceive lower dimension
    \item \textbf{Low-energy probes} (large $t$): Probe long-distance structure, recover classical dimension
\end{itemize}

This aligns with the intuitive picture of quantum gravity: ``At Planck scale, spacetime behaves like 2D; at macroscopic scale, spacetime behaves like 4D.''

\subsection{Dimension Flow}

\textbf{Dimension flow} describes the evolution of spectral dimension with energy scale:
\begin{equation}
    d_s = d_s(\mu), \quad \mu \in \mathbb{R}^+
\end{equation}

Expected behavior:
\begin{align}
    d_s(\mu) &\to 2 \text{ as } \mu \to \infty \text{ (UV)}\\
    d_s(\mu) &\to 4 \text{ as } \mu \to 0 \text{ (IR)}
\end{align}

This flow provides a new perspective on understanding quantum gravity: \emph{dimension is not a fixed background, but a dynamical result of physics}.

\section{Contributions of This Work}\label{sec:contributions}

This paper establishes \textbf{Dimensionics-Physics}: a mathematically rigorous framework for dimension theory. Our contributions include:

\subsection{Theoretical Foundation}
\begin{itemize}
    \item \textbf{9 axioms} (A1-A9) defining the mathematical structure
    \item \textbf{12 theorems} with rigorous proofs (L1 strictness)
    \item \textbf{Master Equation}: $\mu \frac{\partial d_s}{\partial \mu} = \beta(d_s)$ governing dimension evolution
    \item Connection to Fixed-4D-Topology framework
\end{itemize}

\subsection{Physical Results}

\begin{theorem}[Modified Relativity]\label{thm:effective-metric}
The effective metric is $g^{\text{eff}}_{\mu\nu} = \frac{4}{d_s} g_{\mu\nu}$, with modified Lorentz group $SO(3,1; d_s)$.
\end{theorem}

\begin{theorem}[UV Fixed Point]\label{thm:uv-fixed}
$\lim_{\mu \to \infty} d_s = 2$ with power-law convergence.
\end{theorem}

\begin{theorem}[Black Hole Dimension]\label{thm:bh-dim}
Near a Schwarzschild black hole: $d_s(r) = 4 - \frac{r_s}{r}$.
\end{theorem}

\begin{theorem}[Cosmic Evolution]\label{thm:cosmic}
$d_s(t) = 2 + \frac{2}{1 + e^{-(t-t_c)/\tau}}$.
\end{theorem}

\subsection{Experimental Predictions}

We derive \textbf{11 experimental predictions}, including:

\begin{itemize}
    \item \textbf{P1 (CMB)}: $C_\ell = C_\ell^{\Lambda\text{CDM}} \cdot (\ell/\ell_*)^{4-d_s}$, testable by CMB-S4
    \item \textbf{P2 (GW)}: $\omega^2 = c^2 k^2 [1 + \frac{\beta_0}{2}(E/E_{\text{Pl}})^\alpha]$, accessible to LISA
    \item \textbf{P4, P8, P9, P11}: Already verified (percolation, networks, spin chains, critical exponents)
\end{itemize}

\subsection{Relation to M-Series}

This work is independent of the M-1 through M-10 series. While M-1's methodological ideas were influential, all mathematical definitions, theorem proofs, and physical predictions were derived independently within the Fixed-4D-Topology framework. See Appendix~\ref{app:mseries} for detailed comparison.

\section{Structure of This Paper}

\begin{itemize}
    \item \textbf{Chapter~\ref{sec:axioms-system}}: Axiomatic foundation (A1-A9)
    \item \textbf{Chapter~\ref{sec:master-rg}}: Dimension flow and RG analysis
    \item \textbf{Chapter~\ref{sec:eff-metric}}: Modified relativity and P2
    \item \textbf{Chapter~\ref{sec:uv-fixed}}: Quantum gravity applications
    \item \textbf{Chapter~\ref{sec:cosmic-evolution}}: Cosmology and P1
    \item \textbf{Chapter~\ref{sec:major-predictions}}: Experimental predictions summary
    \item \textbf{Chapter~\ref{sec:synthetic}}: Comparison with other QG theories
    \item \textbf{Chapter~\ref{sec:summary}}: Conclusion and outlook
\end{itemize}
